% Preamble
\documentclass[11pt]{article}

% Packages
\usepackage{amsmath}
\usepackage{amssymb}
\usepackage{siunitx}
\usepackage{physics}
\usepackage{todonotes}
\usepackage{graphicx}
\usepackage{url}
\usepackage[plain]{fancyref}
\usepackage[
style=phys, articletitle=false, biblabel=brackets, chaptertitle=false, pageranges=false, url=true
]{biblatex}
\addbibresource{no_cutoff.bib}

% Document
% !TeX spellcheck = en_GB
\begin{document}

	\graphicspath{{figures/}}

	\section{Very large cutoff} \label{sec:intro}

	All these values calculated using the interpolated Nam form obtained earlier (and ultimately deriving from~\cite{Nam1967}).
	The Lindhard forms are essentially from~\cite{SolyomV3}, although they are again simplified in the appropriate ranges.

	Here we look at the case where we get rid of the cutoff's effects and push them outside our range of values.
	In \fref{fig:t1ez}, we have a very large range of cutoffs to show that the precise value doesn't really affect $T_1$ for this range of $z$.
	This is probably unphysically large;
	however from previous discussions that might not matter.

	\begin{figure}[htp]
		\centering
		\includegraphics[width=12cm]{T1ZE1}
		\caption{$T_{1}^{(E)}(z)$, with $z$ in units of $\flatfrac{\omega}{c}$ and $T_1$ in seconds.
		Three different cutoff values shown, from $10^{15}$ through $10^{25}$, to show that in this distance range the cutoff is stable}\label{fig:t1ez}
	\end{figure}

	In \fref{fig:t2ez} we have the value with different temperatures.
	This has the expected form;
	as temperature gets closer to $T_c$, $T_1$ approaches the normal metal value.

	\begin{figure}[htp]
		\centering
		\includegraphics[width=12cm]{T1ZE2}
		\caption{$T_{1}^{(E)}(z)$, for different temperatures.
			In the legend $T$ is given in units of $T_c$.
			For the Lindhard case, temperature matters less than the three superconducting cases.
		} \label{fig:t2ez}
	\end{figure}

	In \fref{fig:t3ez} we vary the frequency.
	Again, this seems to operate as expected, with higher frequencies better able to excite quasiparticles in the SC and decreasing $T_1$.
	The graph in \fref{fig:t5ez} is similar.
	There, we have two separate values of $\omega$, and for each both the Lindhard and Nam forms are plotted.
	This helps us see how much of the variation with $\omega$ comes from superconducting specific effects.

	\begin{figure}[htp]
		\centering
		\includegraphics[width=12cm]{T1ZE3}
		\caption{
			$T_{1}^{(E)}(z)$, for different frequencies.
			All are Nam calculations.
		} \label{fig:t3ez}
	\end{figure}

	\begin{figure}[htp]
		\centering
		\includegraphics[width=12cm]{T1ZE5}
		\caption{
			Comparing Nam and Lindhard each at two frequencies.
		} \label{fig:t5ez}
	\end{figure}

	\Fref{fig:t4ez} varies $\tau$.
	The shape of the dependence is different from the $\omega$ dependence, which is interesting.\todo{Add Lindhard comparison curves}
	It seems that $\tau$ matters less for the noise at further distances.\todo{Why might that be?}

	\begin{figure}[htp]
		\centering
		\includegraphics[width=12cm]{T1ZE4}
		\caption{
			$T_{1}^{(E)}(z)$, for different $\tau$.
			All are Nam forms.
		} \label{fig:t4ez}
	\end{figure}

	\newpage
	\listoftodos
	\newpage
	\printbibliography

\end{document}
