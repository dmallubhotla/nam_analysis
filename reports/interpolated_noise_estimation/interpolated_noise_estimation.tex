% Preamble
\documentclass[11pt]{article}

% Packages
\usepackage{amsmath}
\usepackage{amssymb}
\usepackage{siunitx}
\usepackage{physics}
\usepackage{todonotes}
\usepackage{graphicx}
\usepackage{url}
\usepackage[plain]{fancyref}
\usepackage[
	style=phys, articletitle=false, biblabel=brackets, chaptertitle=false, pageranges=false, url=true
]{biblatex}
\addbibresource{interpolated_noise_estimation.bib}

% Document
% !TeX spellcheck = en_GB
\begin{document}

	\graphicspath{{figures/}}

	\section{Interpolated dielectric constant} \label{sec:intro}

	As discussed earlier, we can try to get a handle on some of the qualitative features of the noise integration by using a piecewise dielectric constant that closely approximates the value in~\cite{Nam1967}.
	We incorporate a cutoff of the imaginary part above some $u_{uc}$ on the order of the $u_F$, which is $\sim 10^{10}$ for the materials we might consider.
	\begin{equation}
		\epsilon(u, u_{uc}) =
		\begin{cases}
			-a + i b & u < u_c \\
			1 + \frac{-c + i d}{u} &  u_{uc} > u \geq u_c \\
			1 & u > u_{uc}
		\end{cases}, \label{eq:eps}.
	\end{equation}
	where $a$, $b$, $c$ and $d$ are real and positive (and are determined by the frequency) and $u_c$ is a cutoff value on the order of the impurity spacing.
	Approximating this is fairly straightforward, as the integrals are stable for small $u$ in the full conductivity expression, allowing $a$ and $b$ to be determined without issue.
	We previously implemented Nam's expression numerically in a Mathematica package in terms of the expression $\Sigma(\xi, \kappa, \nu, t)$, which were dimensionless quantities related to the gap $\Delta$.
	This $\Sigma$ is related to the dielectric constant via
	\begin{equation}
		\epsilon = 1 + 4 \pi i \frac{\sigma_N}{\omega} \Sigma.
	\end{equation}
	Here, $\sigma_N$ was related to the conductivity of the metal in the normal state, which could be calculated, but we took simply as an external parameter.

	The recipe then is straightforward to derive the values of the constants in~\eqref{eq:eps}:
	\begin{enumerate}
		\item Use $T$ and $T_c$ to find a value for $\Delta$, which I did using the expression $\Delta = 3.06 \sqrt{T_c(T_c - T)}$.
		\item Find the values $\xi = \flatfrac{\omega}{\Delta}$, $\nu = \flatfrac{1}{(\tau \Delta)}$,  $t = \flatfrac{T}{\Delta}$ and $A = \flatfrac{\omega v_F}{(c \Delta)}$.
		\item Generally, $\epsilon(u) = 1 + 4 \pi i \frac{\sigma_N}{\omega} \Sigma(\xi, A u, \tau, t)$.
			Use the asymptotic expression for $\Sigma$ as $u \rightarrow 0$.
			Then,
			\begin{equation}
				-a + ib = 4 \pi i \frac{\sigma_N}{\omega} \Sigma(A u \rightarrow 0),
			\end{equation}
			from which $a$ and $b$ can be extracted by taking the real and imaginary parts.
		\item Similarly, for large $\kappa = A u$,
			\begin{align}
				\frac{-c + i d}{u} &= 4 \pi i \frac{\sigma_N}{\omega} \Sigma(\kappa > u_c) \\
				- c + i d &= 4 \pi i \frac{\sigma_N}{\omega} \Sigma(\kappa > u_c) u \\
				- c + i d &= 4 \pi i \frac{\sigma_N}{\omega} \Sigma(\kappa > u_c) \frac{\kappa}{A}
			\end{align}
			If we want to determine $c$ and $d$, we can simply increase $\kappa$ until this expression stops varying greatly, which in practice will be determined by $\tau$ and $v_F$ (as should be expected).
		\item We can determine an interpolation constant by finding $u_c$ such that
			\begin{align}
				-a + ib &= 1 + \frac{-c + id}{u_c} \\
				u_c &= \frac{-c + id}{-(a + 1) + ib}.
			\end{align}
			This is allowed to be complex, but in practice $u_c$ has a much larger real part, so we can simplify our code by simply defining $u_c$ with a real part as well.
			This is around where~\eqref{eq:eps} is its worst approximation, and presumably we could make this more accurate in the future.
	\end{enumerate}

	This procedure lets us use the packages already developed to determine $\epsilon$ as a simple piecewise form.
	There doesn't seem to be a clear way to determine $u_{uc}$ at this stage, and as we saw in the earlier notes, small variations in that cutoff parameter make a large difference in the final noise calculations.


	\newpage
	\listoftodos
	\newpage
	\printbibliography

\end{document}
