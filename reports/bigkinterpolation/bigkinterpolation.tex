% Preamble
\documentclass[11pt]{article}

% Packages
\usepackage{amsmath}
\usepackage{amssymb}
\usepackage{siunitx}
\usepackage{physics}
\usepackage{todonotes}
\usepackage{graphicx}
\usepackage{url}
\usepackage[plain]{fancyref}
\usepackage[
	style=phys, articletitle=false, biblabel=brackets, chaptertitle=false, pageranges=false, url=true
]{biblatex}
\addbibresource{bigkinterpolation.bib}

% Document
% !TeX spellcheck = en_GB
\begin{document}

	\graphicspath{{figures/}}

	\section{Interpolated dielectric constant} \label{sec:bigkinterpolation}

	As we saw in the earlier notes on the noise values, the $k$ dependence of the response function $\epsilon_{AN}$(the Approximated Nam) in~\cite{Nam1967} are fairly well approximated with a piecewise function:
	\begin{equation}
		\epsilon_{AN}(u) =
		\begin{cases}
			-a + i b & u < u_c \\
			1 + \frac{-c + i d}{u} & u \geq u_c
		\end{cases}, \label{eq:epsAN}
	\end{equation}
	where $a$, $b$, $c$ and $d$ are real and positive (and are determined by the frequency) and $u_c$ is a cutoff value on the order of the impurity spacing.
	This approximates Nam's $\epsilon$ well for $u \ll u_c$ and $u \gg u_c$, and is somewhat less accurate very close to $u_c$.
	For the values mentioned in the other notes for lead, fitting the numerical integration gives
	\begin{align}
		a &= 3.81 \times 10^{11} \\
		b &= 3.24 \times 10^{9} \\
		c &= 2.67 \times 10^{19} \\
		d &= 2.29 \times 10^{17} \\
		u_c &= 7 \times 10^{7}
	\end{align}.

	This is all well below the Fermi wave vector, which in these units is on the order of $u_F \sim 10^{15}$.
	We should expect that at some upper cutoff $u_{uc} \lesssim u_F$, the SC effects die out.
	In order to incorporate that, we might consider forcing the dielectric constant to manually match the normal metal above some value.
	For a normal metal, for sufficiently large $u$,
	\begin{equation}
		\epsilon_{TF} = 1 + \flatfrac{u_{TF}^2}{u^2} \label{eq:epsTF},
	\end{equation}
	where $u_{TF}$ is of the same order as $u_F$.\todo{cite this}
	We would expect~\eqref{eq:epsAN} to fail at some point approaching $u_{uc}$, and at that scale it's numerically clear that we won't be able to smoothly join this with~\eqref{eq:epsTF}.
	The salient features of this expansion are that above some critical $u$ the imaginary part of $\epsilon$ goes to zero, so we might consider forcing $\epsilon$ above $u_{uc}$ to follow
	\begin{equation}
		\epsilon(u > u_{uc}) = 1 + \frac{-c u_{uc}}{u^2}, \label{eq:epsSmoothJoin}
	\end{equation}
	which will smoothly join with~\eqref{eq:epsAN} and replicates the $u^{-2}$ dependence for the real part of $\epsilon$.
	Alternatively, we can just use a piecewise form
	\begin{equation}
		\epsilon_f(u) =
		\begin{cases}
			-a + i b & u < u_c \\
			1 + \frac{-c + i d}{u} &  u_{uc} \geq u \geq u_c \\
			1 + \frac{u_{TF}^2}{u} &  u \geq u_{uc}
		\end{cases}. \label{eq:epsForced}
	\end{equation}
	It turns out that both options are reasonably similar when going through the surface impedance calculations, so I'll use~\eqref{eq:epsForced} for these notes.

	\section{Plots of noise integrand} \label{sec:plotsdiscussion}

	For all of these values, above $u > u_{uc}$ the value of $\Im r_p(u)$ becomes identically zero.
	This does make the function converge, however, there isn't unfortunately a natural choice for $u_{uc}$.
	\begin{figure}[htp]
		\centering
		\includegraphics[width=12cm]{imrp1}
		\caption{$\Im[r_p(u)]$ for different cutoff $u_{uc}$} \label{fig:imrpVsCutoff}
	\end{figure}

	We see \fref{fig:imrpVsCutoff} the effect of various cutoff values.
	We would expect that $u_{uc}$, should be less than the Fermi momentum, but we see that even slight changes have a big impact in the actual integrand.
	For an SC, it might make sense instead to put the limit based on the Debye momentum, which is much less than the Fermi momentum.
	As discussed elsewhere, the noise should converge for $z \rightarrow 0$, and this corresponds to
	\begin{equation}
		\chi_{EE}(z = 0) \propto \int_0^\infty \dd{u} u^2 Im[r_p(u)].
	\end{equation}

	\begin{figure}[htp]
		\centering
		\includegraphics[width=12cm]{imrp2}
		\caption{$u^2 \Im[r_p(u)]$ for different cutoff $u_{uc}$} \label{fig:u2imrpVsCutoff}
	\end{figure}
	The integrand, $u^2 \Im[r_p(u)]$, is plotted in \fref{fig:u2imrpVsCutoff}.
	Unfortunately, it's clear that the value of the integral is strongly dependent on the specific value of the cutoff $u_{uc}$, which makes it hard to find a natural value using heuristic arguments.

	As mentioned above, the specific form of the cutoff turns out to be unimportant, as the shape of $\Im r_p(u)$ is largely governed by the parameter $c$.
	For example, using
	\begin{equation}
		\epsilon_f(u) =
		\begin{cases}
			-a + i b & u < u_c \\
			1 + \frac{-c + i d}{u} &  u_{uc} \geq u \geq u_c \\
			1                      &  u \geq u_{uc}
		\end{cases} \label{eq:epsForcedOne}
	\end{equation}
	instead of~\eqref{eq:epsForced} produces the plot shown in \fref{fig:u2imrpVsCutoffNoResponseAbove}, which is effectively the same as \fref{fig:u2imrpVsCutoff}.
	Similarly, using \fref{eq:epsSmoothJoin} produces \fref{fig:u2imrpVsCutoffSmoothJoin}, which is again the same, justifying the assertion that the specific form above our cutoff is irrelevant as long as the imaginary part is zero.
	\begin{figure}[htp]
		\centering
		\includegraphics[width=12cm]{imrpU2NoResponseAboveCutoff}
		\caption{$u^2 \Im[r_p(u)]$ for different cutoff $u_{uc}$, using~\eqref{eq:epsForcedOne}.} \label{fig:u2imrpVsCutoffNoResponseAbove}
	\end{figure}

	\begin{figure}[htp]
		\centering
		\includegraphics[width=12cm]{imrpU2SmoothJoin}
		\caption{$u^2 \Im[r_p(u)]$ for different cutoff $u_{uc}$, using~\eqref{eq:epsSmoothJoin}.} \label{fig:u2imrpVsCutoffSmoothJoin}
	\end{figure}

	\section{Noise calculations} \label{sec:noise}
	\begin{figure}[htp]
		\centering
		\includegraphics[width=12cm]{chiZEsmallboi}
		\caption{$\chi_{EE}(z)$, with $z$ in units of $\flatfrac{\omega}{c}$, for different cutoffs} \label{fig:chiZRange}
	\end{figure}
	As we might expect from the shape of $\Im r_p$, this naive approach ends up with the noise strongly depending on the cutoff value.
	We do see in \fref{fig:chiZRange} that the $\flatfrac{1}{z^3}$ dependence goes to a constant value for $z \lesssim \frac{1}{u_{uc}}$.
	This is unsurprising, because we engineered the form of $\epsilon$ to specifically manufacture that convergence.



	\newpage
	\listoftodos
	\newpage
	\printbibliography

\end{document}
