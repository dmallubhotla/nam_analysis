% Preamble
\documentclass[11pt]{article}

% Packages
\usepackage{amsmath}
\usepackage{physics}
\usepackage{todonotes}

% Document
\begin{document}

\section{Nam form of conductivity} \label{sec:NamForm}

Notes on the Nam conductivity function.
We'll want to do the calculations dimensionlessly.
To remove units, we'll want to represent all the various quantities in terms of $\Delta$:
\begin{align}
	\xi &= \frac{\omega}{\Delta} \\
	\xi' &= \frac{\omega'}{\Delta} \\
	\nu &= \frac{1}{\tau \Delta} \\
	\kappa &= \frac{q v_0}{\Delta} \\
	t &= \frac{T}{\Delta} \\
	\sigma_0 &= \frac{n e^2}{m \Delta}
\end{align}

From Nam, we have
\begin{equation}
	\sigma(\kappa, \xi) = -i \frac{3 \sigma_0}{4} \frac{1}{\xi}\left[\int_{1 - \xi}^{1}\dd{\xi} \tanh(\frac{\xi + \xi'}{2 t}) I_1 + \int_{1}^{\infty} \dd{\xi'} \left( \tanh(\frac{\xi + \xi'}{2t}) I_1  - \tanh(\frac{\xi'}{2t})I_2 \right) \right]
\end{equation}
with
\begin{align}
	I_1 &= F(\kappa, \Re[\sqrt{(\xi + \xi')^2 - 1} - \sqrt{\xi'^2 - 1}]) (g + 1) \nonumber\\
	&\quad + F(\kappa, \Re[-\sqrt{(\xi + \xi')^2 - 1} - \sqrt{\xi'^2 - 1}]) (g - 1) \\
	I_2 &= F(\kappa, \Re[\sqrt{(\xi + \xi')^2 - 1} - \sqrt{\xi'^2 - 1}]) (g + 1) \nonumber\\
	&\quad + F(\kappa, \Re[\sqrt{(\xi + \xi')^2 - 1} + \sqrt{\xi'^2 - 1}]) (g - 1) \\
	F(\kappa, E) &= \frac{1}{\kappa} \left[2 S(E) + (1 - S(E)^2)\ln(\frac{S(E) + 1}{S(E) - 1})\right]  \\
	S(\kappa, E) &= \frac{1}{\kappa} \left(E - i \left(\Im[\sqrt{(\xi + \xi')^2 - 1} + \sqrt{\xi'^2 - 1}] + 2 \nu \right) \right) \\
	g  &= \frac{\xi' \left( \xi + \xi'\right) + 1}{\sqrt{\xi'^2 - 1}\sqrt{(\xi + \xi')^2 - 1}}
\end{align}

\subsection{Verifying small $\kappa$ dependence} \label{subsec:NamForm.SmallK}
We should expect that the conductivity reaches a finite value as $\kappa \rightarrow 0$.
To verify this, we'll want to actually take that limit.
All of the dependence on momentum comes in through the function $F$ and $S$, so we can begin by writing $S$ as $S = \frac{\eta}{\kappa}$, which means that
\begin{align}
	F &= \frac{1}{\kappa} \left[2 S(E) + (1 - S(E)^2)\ln(\frac{S(E) + 1}{S(E) - 1})\right] \\
	F &= \frac{1}{\kappa} \left[2 \frac{\eta}{\kappa} + (1 - \frac{\eta^2}{\kappa^2})\ln(\frac{\frac{\eta}{\kappa} + 1}{\frac{\eta}{\kappa} - 1})\right] \label{eq:NamForm:FPreLnExpand}
\end{align}

We can then expand out the log term:
\begin{align}
	\ln(\frac{\frac{\eta}{\kappa} + 1}{\frac{\eta}{\kappa} - 1}) &= \ln(\frac{\eta + \kappa}{\eta - \kappa}) \\
	&= 2 \frac{\kappa}{\eta} + \frac23 \left(\frac{\kappa}{\eta}\right)^3 + \frac25 \left(\frac{\kappa}{\eta}\right)^5 + \mathcal{O}\left(\left(\frac{\kappa}{\eta}\right)^7\right)
\end{align}

Plugging the first two terms into~\eqref{eq:NamForm:FPreLnExpand} gives us
\begin{align}
	F &= \frac{1}{\kappa} \left[2 \frac{\eta}{\kappa} + (1 - \frac{\eta^2}{\kappa^2})\ln(\frac{\frac{\eta}{\kappa} + 1}{\frac{\eta}{\kappa} - 1})\right] \\
	&= \frac{1}{\kappa} \left[2 \frac{\eta}{\kappa} + (1 - \frac{\eta^2}{\kappa^2})\left(2 \frac{\kappa}{\eta} + \frac23 \left(\frac{\kappa}{\eta}\right)^3\right)\right] \\
	&= \frac{1}{\kappa} \left[2 \frac{\eta}{\kappa} + 2\frac{\kappa}{\eta} - 2 \frac{\eta}{\kappa} + \frac23 \frac{\kappa^3}{\eta^3} - \frac23 \frac{\kappa}{\eta}\right] \\
	&= \frac{1}{\kappa} \left[\frac43\frac{\kappa}{\eta}\right] \\
	&= \frac43 \frac{1}{\eta}
\end{align}
Here we dropped the second leading term in $\kappa^3$ before simplifying, to find that $F$ does indeed approximate a constant value.

\todo{Everything onwards needs serious editing}
\section{Graphing the conductivity} \label{sec:NamForm.Graph}

One way we can verify that our code behaves as desired is to check that graphs of the conductivity show features that we would expect.
For all of our graphs, the easiest to visualise is the impurity parameter.
For a longer mean free path, we can see a higher conductivity, as expected.
Like was mentioned above, for graphs of $\Sigma$ we won't see the full dependence on the collision frequency, because normal conductivity depends on the scattering time as well.
To keep the visualisations cleaner, we'll fix the collision frequency here and see how $\nu = .1$ compares to $\nu = 10$ at most.\todo{Include a couple graphs showing that $\nu$ isn't actually important}

\subsection{$\xi$ dependence} \label{subsec:NamForm.Graph.OmegaDep}
To begin with, we can look at the simplest case for $\kappa = 0$ and $t = 0$. \todo{Include t = 0 frequency = 0 graphs, real and imaginary}
As expected, we see that for $\xi < 2$ that there is no real conductivity. This makes sense; there are no available states in the gap for conductivity to occur. \todo{maybe explicitly show where this comes from in the Nam equation?}
For non-zero temperatures, we see that there are low energy excitations, leading to a cusp in the graph and a non-zero real part in the gap. \todo{Include nonzero T vs frequency graphs}

Also, for $\kappa \rightarrow 0$ we expect that we should be in the local limit.
This was explored in the Zimmermann paper, and we can compare that our graphs here for $\kappa = 0$ have the right form. \todo{Include ref to zimmermann paper and make sure graphs match}

\subsection{$\kappa$ dependence} \label{subsec:NamForm.Graph.KDep}
As expected, we can see that the behaviour as a function of $\kappa$ is as expected, with the conductivity becoming real as $\kappa \rightarrow 0$, but going as $\frac{1}{\kappa}$ for large $\kappa$. \todo{Include graphs against kappa both for small frequency high temp and high frequency low temp (and I guess high frequency high temp if desired}


\subsection{$t$ dependence} \label{subsec:NamForm.Graph.TDep}
Similarly, we can show the graph here as a function of $t$. \todo{Show graphs vs reduced temp}
Because we are making things dimensionless as a function of $\Delta$, which itself depends on temperature, it may also be helpful to write $\Delta = f(T, T_c)$\todo{get real equation} and graph with the actual temperature dependence induced by $\Delta$. \todo{Include those graphs with actual temp dependence of energy gap and temperature dependent scale}

\section{Signs of conductivity real and imaginary part} \label{sec:NamForm.Signs}
From the graphs above, we can see the signs obtained for real and imaginary parts of $\Sigma$. \todo{Find some nice arguments showing exactly what sign the conductivity needs to have in particular regions and make sure that graph replicates them. And make sure to work out explicitly what Nam predicts for sign.}
There are some places where the sign must be handled carefully; the code used to generate the plots above differ from Nam in a couple places \todo{include notation and notes that show where and how they differ}.

\end{document}
