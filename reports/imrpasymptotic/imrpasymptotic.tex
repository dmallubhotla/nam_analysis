% Preamble
\documentclass[11pt]{article}

% Packages
\usepackage{amsmath}
\usepackage{amssymb}
\usepackage{siunitx}
\usepackage{physics}
\usepackage{todonotes}
\usepackage{graphicx}
\usepackage{url}
\usepackage[plain]{fancyref}
\usepackage[
	style=phys, articletitle=false, biblabel=brackets, chaptertitle=false, pageranges=false, url=true
]{biblatex}
\addbibresource{imrpasymptotic.bib}

% Document
% !TeX spellcheck = en_GB
\begin{document}

	\section{$\Im r_p(u)$ large-$u$ description} \label{sec:imrpasymptotic}

	We can start by looking at our description of $\epsilon(u)$.
	Unlike for the standard metal case, $\epsilon$ decays as $\frac{1}{u}$, rather than $\frac{1}{u^2}$, which changes the convergence characteristics of $\zeta_p$ and $r_p$.
	From \cite{Churchill2016}, we would hope that we could avoid deliberately cutting off any integrals at a particular lattice constant, so we can start by assuming the integrals are all to $\infty$.


	\newpage
	\listoftodos
	\newpage
	\printbibliography

\end{document}
