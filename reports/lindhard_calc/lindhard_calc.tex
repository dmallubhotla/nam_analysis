% Preamble
\documentclass[11pt]{article}

% Packages
\usepackage{amsmath}
\usepackage{amssymb}
\usepackage{siunitx}
\usepackage{physics}
\usepackage{todonotes}
\usepackage{graphicx}
\usepackage{url}
\usepackage[plain]{fancyref}
\usepackage[
style=phys, articletitle=false, biblabel=brackets, chaptertitle=false, pageranges=false, url=true
]{biblatex}
\addbibresource{lindhard_calc.bib}

% Document
% !TeX spellcheck = en_GB
\begin{document}

	\graphicspath{{figures/}}

	\section{Noise calculation using Lindhard function} \label{sec:intro}

	Let's calculate the noise using the Lindhard dielectric function.
	Solyom\supercite{SolyomV3} has, for finite scattering times, the dielectric function as
	\begin{equation}
		% explicit \displaystyle here for fraction
		\epsilon(\vec{q}, \omega) = 1 + \frac{q_{TF}^2}{q^2}\frac{\displaystyle 1 + \frac{\omega + \flatfrac{i}{\tau}}{2 v_F q} \ln(\frac{\omega - v_F q + \flatfrac{i}{\tau}}{\omega + v_F q + \flatfrac{i}{\tau}})}{\displaystyle 1 + \frac{\flatfrac{i}{\tau}}{2 v_F q} \ln(\frac{\omega - v_F q + \flatfrac{i}{\tau}}{\omega + v_F q + \flatfrac{i}{\tau}})} \label{eq:lindhardsolyom}
	\end{equation}

	For later, we have $q_{TF}^2 = 4 \pi e^2 \rho(e_F)$, where $e$ is the renormalised charge.
	With $\omega_p^2 = \flatfrac{4 \pi n e^2}{m}$ and $q_{TF}^2 = \flatfrac{3 \omega_p^2}{v_F^2}$, we can write
	\begin{align}
		q_{TF}^2 &= 4 \pi e^2 \rho(e_F) \\
		\frac{3 \omega_p^2}{v_F^2} &= 4 \pi e^2 \rho(e_F) \\
		\frac{3 \flatfrac{4 \pi n e^2}{m}}{v_F^2} &= 4 \pi e^2 \rho(e_F) \\
		\frac{3n}{v_F^2 m} &= \rho(e_F) \\
	\end{align}
	To synchronise constants:
	\begin{align}
		\sigma_{DC} &= \frac{n e^2 \tau}{m} \\
		&= \frac{1}{4\pi}\frac{4 \pi n e^2}{m} \tau \\
		&= \frac{n e^2 \tau}{m} \\
		&= \frac{\omega_p^2}{4\pi}\tau
	\end{align}
	In the noise notes we had $\sigma_{DC} = \SI{10e16}{\per\second}$ and $\tau = \SI{10e-14}{\per\second}$
	\begin{align}
		10^{16} &= \frac{\omega_p^2}{4\pi} 10^{-14}\\
		\omega_p^2 &= 4 \pi \times 10^{30} \\
		\omega_p &\approx \SI{3.5e15}{\per\second}
	\end{align}
	This is a reasonable value for $\omega_p$, which reassures us that our previous constants were reasonable as well.
	With $v_F = \SI{2e6}{\m \per \s}$, we can write
	\begin{align}
		q_{TF}^2 &= \frac{3 \omega_p^2}{v_F^2} \\
		&= \frac{12\pi \times 10^{30}}{\left( 2 \times 10^6 \right)^2} \\
		q_{TF} &\approx \SI{3e9}{\per\m}
	\end{align}
	Along with $\omega = \SI{10e9}{\per\second}$, this sets all the constants we would use for calculation in~\eqref{eq:lindhardsolyom}.
	\section{Electric noise} \label{sec:ElectricNoise}
	In \fref{fig:chizee} we can see the electric noise, without units.
	\begin{figure}[htp]
		\centering
		\includegraphics[width=12cm]{chiZELindhard}
		\caption{$\chi_{EE}(z)$, with $z$ in units of $\flatfrac{\omega}{c}$} \label{fig:chizee}
	\end{figure}
	We multiply our noise by
	\begin{align}
		\chi = \frac{\hbar}{\epsilon_0} I
	\end{align}

	The results of $T_1^{(E)}(z)$ are plotted in \fref{fig:t1ez}, although the units seem off.
	This is calculated via
	\begin{equation}
		\frac{1}{T_1^{(E)}}
	\end{equation}
	\begin{figure}[htp]
		\centering
		\includegraphics[width=12cm]{T1ZELindhard}
		\caption{$T_{1}^{(E)}(z)$, with $z$ in units of $\flatfrac{\omega}{c}$} \label{fig:t1ez}
	\end{figure}

	\newpage
	\listoftodos
	\newpage
	\printbibliography

\end{document}
