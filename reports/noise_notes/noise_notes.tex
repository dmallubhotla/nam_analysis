% Preamble
\documentclass[11pt]{article}

% Packages
\usepackage{amsmath}
\usepackage{amssymb}
\usepackage{siunitx}
\usepackage{physics}
\usepackage{todonotes}
\usepackage{graphicx}
\usepackage{url}
\usepackage[plain]{fancyref}
\usepackage[
	style=phys, articletitle=false, biblabel=brackets, chaptertitle=false, pageranges=false, url=true
]{biblatex}
\addbibresource{noise_notes.bib}

% Document
% !TeX spellcheck = en_GB
\begin{document}

	\graphicspath{{figures/}}

	\section{Noise notes} \label{sec:NoiseCalc}
	An excellent review of the calculation of relaxation rates is available in~\cite{Henkel1999}.
	The theory behind the noise calculations comes in large part from~\cite{Agarwal1975}.

	\subsection{Noise calc results} \label{subsec:NoiseCalcResults}
	In~\cite{Henkel1999}, the authors find that the noise depends on the spectral field density.
	They consider a system with an infinite conducting plane a distance $z$ away from a trap;
	the geometry here affects the Green functions that they use, so these results are specific to that geometry.
	Starting with the spectral density of a blackbody field, $S^{bb}_E(\omega)$, they have
	\begin{equation}
		S^{(bb)}_E(\omega) = \frac{\hbar \omega^3}{3 \pi \epsilon_0 c^3 (1 - e^{-\flatfrac{\hbar \omega}{T}})}
	\end{equation}
	They then relate this to the near field noise expression:
	\begin{equation}
		S^{(nf)ij}_E(z, \omega) = S^{(bb)}_E(\omega) g_{ij}(z),
	\end{equation}
	where $z$ is in units of the vacuum wavelength $\frac{c}{omega}$ and $g_{ij}$ is a dimensionless noise tensor with components in the directions parallel ($g_{xx}$ and $g_{yy}$) and perpendicular ($g_{zz}$) to the surface as follows:
	\begin{gather}
		g_\parallel = \frac34 \Re \int_0^{+\infty} \dd{u} \frac{u}{v} e^{2i z v} \left(r_s(u) + \left(u^2 - 1\right) r_p(u) \right) \\
		g_\perp = \frac32 \Re \int_0^{+\infty} \dd{u} \frac{u^3}{v} e^{2 i z v} r_p(u)
	\end{gather}
	Here $v = \sqrt{1 - u^2}$, and we take the root $v = i \sqrt{u^2 - 1}$ for $u \geq 1$.
	This notation seems preferable over incorporating factors of $\frac{\omega}{c}$, because we can keep everything dimensionless, and it emphasises that the only dependence on $z$ or $\omega$ is via $z$.

	The magnetic noise tensors $h_\parallel$ and $h_\perp$ are exactly the same, except swapping $r_s \leftrightarrow r_p$ and with an extra factor of $\frac{1}{c^2}$:
	\begin{gather}
		S^{(nf)ij}_B(z, \omega) = \frac{S^{(bb)}_E(\omega)}{c^2} h_{ij}(z) \\
		h_\parallel = \frac34 \Re \int_0^{+\infty} \dd{u} \frac{u}{v} e^{2i z v} \left(r_p(u) + \left(u^2 - 1\right) r_s(u) \right) \\
		h_\perp = \frac32 \Re \int_0^{+\infty} \dd{u} \frac{u^3}{v} e^{2 i z v} r_s(u)
	\end{gather}

	\subsection{Reflection coefficients} \label{subsec:ReflectionCoeffs}
	This shows that the only dependence on the properties of the conducting material comes in through the reflection coefficients $r_p$ and $r_s$.
	The standard Fresnel coefficients defined by
	\begin{align}
		r_p(u) = \frac{\epsilon v - \sqrt{\epsilon - u^2}}{\epsilon v + \sqrt{\epsilon - u^2}} \\
		r_s(u) = \frac{v - \sqrt{\epsilon - u^2}}{v + \sqrt{\epsilon - u^2}}
	\end{align}
	(with relative dielectric constant $\epsilon$) can serve as a local equation\todo{add plots showing what the actual graphs of the noise look like with the Fresnel reflection coefficients};
	they can, however, be extended by calculating the surface impedances as described by~\cite{Ford1984} (with similar results appearing in~\cite{Nam1967_Part2}).
	They have, for cases where the initial medium has a response with $\epsilon = 1$\todo{rephrase},
	\begin{align}
		r_p(u) &= \frac{\pi v - \zeta_p(u)}{\pi v + \zeta_p(u)} \\
		r_s(u) &= \frac{\zeta_s(u) - \frac{\pi}{v}}{\zeta_s(u) + \frac{\pi}{v}} \\
		\zeta_p(u) &= 2i \int_0^\infty \dd{y} \frac{1}{\kappa^2} \left( \frac{y^2}{\epsilon_t(\frac{\omega}{c}\kappa, \omega) - \kappa^2} + \frac{u^2}{\epsilon_\ell(\frac{\omega}{c}\kappa, \omega)} \right) \\
		\zeta_s(u) &= 2i \int_0^\infty \dd{y} \frac{1}{\epsilon_t(\frac{\omega}{c}\kappa, \omega) - \kappa^2} \\
		\kappa^2 &= u^2 + y^2
	\end{align}
	To calculate this for SCs, we can calculate the conductivity $\sigma$ and use (noting that the arguments of these functions are no longer dimensionless) $\epsilon(k, \omega) = 1 + i \frac{4 \pi}{\omega} \sigma(k, \omega)$%\todo{is this actually true in this context, or do we need to show that it's actually true in the case here for SCs where we have a nonlocal response?}.

	\section{Sample calculations} \label{sec:SampleCalcs}

	\subsection{Good conductor Fresnel noise} \label{subsec:sample:fresnelperfect}

	As a sanity check we can look at the case of a perfect conductor, with $\epsilon = + i \infty$ (or $\epsilon = 1 + i \infty$, the $1$ doesn't matter).
	This gives us $r_p(u) = 1 + 0 i$ and $r_p(u) = -1 + 0 i$.
	It turns out that the noise integral can be done analytically in that case, giving
	\begin{align}
		g_{\parallel} &= \frac{3}{4} \left(\frac{\sin{2 z}}{4 z^3} - \frac{\cos{2 z}}{2 z^2} - \frac{\sin{2 z}}{z}\right) \\
		g_{\perp} &= \frac32 \left( \frac{\sin{2 z}}{4 z^3} - \frac{\cos{2 z}}{2 z^2} \right)
	\end{align}

	\begin{figure}[htp]
		\centering
		\includegraphics[width=14cm]{SampleElectricNoiseFresnelPerfectConductor}
		\caption{graphs for $\epsilon - 1$, analytic case is $i \infty$} \label{fig:SampleElectricNoiseFresnelPerfect}
	\end{figure}

	\begin{figure}[htp]
		\centering
		\includegraphics[width=14cm]{SampleElectricNoiseZetaPerfectConductor}
		\caption{Using $\zeta$ formulation, graphs for $\epsilon - 1$, analytic case is $i \infty$} \label{fig:SampleElectricNoiseZetaPerfect}
	\end{figure}

	To get a sense of what this looks like, \fref{fig:SampleElectricNoiseFresnelPerfect} shows $g_{\parallel}(z)$, calculated using the Fresnel coefficients with $\epsilon$ with large imaginary parts to represent good conductors.
	We can see that as the conductivity gets larger, the graph more closely represents the analytic solution, but for all finite conductivities the graph diverges for small $z$.
	Typically, the $z$ range of interest is very small: for $z = \SI{1}{\micro\meter}$ and gigahertz frequencies, $z \sim 10^{-5}$.
	This divergent region is unsuitable for the Fresnel equation \todo{Because it diverges, maybe need to clarify?}.

	In \fref{fig:SampleElectricNoiseZetaPerfect} we can see that using the surface impedance forms displays the same relationships for constant $\epsilon$, which is a good sanity check.
	We do run into issues now with nested integrals, which is why it's quicker to create list plots which can be easily parallelised.
	That graph is generated with a Mathematica ListPlot which is then joined\todo{Show mesh or something?}.

	\subsection{Conductivity in region of interest} \label{subsec:sample:namresult}
	We can look at \fref{fig:ConductivityRe} and \fref{fig:ConductivityIm} to try to get an understanding of what $\Sigma$ actually does for physical values.
	Some typical values for the conductivity calculation for a gigahertz device near lead look like $\sigma_0 = 10^7$, $\xi = .01$, $\nu = 700$ and $\kappa ~ .0005$.
	This is a fairly extreme value range for the frequency and momentum, and this leads to some noisiness in the results.
	Comparing this to the other conductivity graphs, we can see that the actual values for $\Sigma$ are very large here, which potentially explains to some extent where the noise actually comes from.

	\begin{figure}[htp]
		\centering
		\includegraphics[width=14cm]{ReT1V700vsKmultW}
		\caption{Real part of Nam conductivity vs $\kappa$, with $\nu = 700$, roughly corresponding to a range for $u$ between $0$ and $10$.} \label{fig:ConductivityRe}
	\end{figure}

	\begin{figure}[htp]
		\centering
		\includegraphics[width=14cm]{ImT1V700vsKmultW}
		\caption{Imaginary part of Nam conductivity vs $\kappa$, with $\nu = 700$, roughly corresponding to a range for $u$ between $0$ and $10$.} \label{fig:ConductivityIm}
	\end{figure}

	There are potentially different strategies for dealing with these results.
	Perhaps the quickest is to note that the values for conductivity drop off quickly, and so some of our integrals can be cut off quickly.\todo{Implement, and see how it looks}
	This could potentially reduce the integral from $u = 0$ to $\infty$ down to a range of $u$ from $0$ to $10$.
	We can also get a sense of the shape of the drop and approximate it with a similarly shaped easily integrable function, and then scale out the large magnitude factor.

	\begin{figure}[htp]
		\centering
		\includegraphics[width=14cm]{SampleElectricNoiseNamZerothOrder}
		\caption{0th order approximation integrating over a constant large $\epsilon$ with $u$-cutoff} \label{fig:NamZerothOrder}
	\end{figure}

	\begin{figure}[htp]
		\centering
		\includegraphics[width=14cm]{SampleElectricNoiseNamZerothOrderGs}
		\caption{0th order approximation for $g_\perp$ integrating over a constant large $\epsilon$ with $u$-cutoff} \label{fig:NamZerothOrderGs}
	\end{figure}

	In \fref{fig:NamZerothOrder}, we have an example of the quick and dirty approximation where we sample one point from the conductivity values above and assume that over some $u$ range of interest (here $0$ to $3$), the $\epsilon$ stays constant.
	Because $\sigma$ is so large, we essentially are in the limit where the delta function peak is extremely far to the left.
	The values for $\omega$ and $c$ mean that $z ~ 10 z$, which means \fref{fig:NamZerothOrder} is graphing noise at about micrometer range.

	\newpage
	\listoftodos
	\newpage
	\printbibliography

\end{document}
