% Preamble
\documentclass[11pt]{article}

% Packages
\usepackage{amsmath}
\usepackage{amssymb}
\usepackage{physics}
\usepackage{todonotes}
\usepackage{graphicx}
\usepackage{url}
\usepackage[plain]{fancyref}
\usepackage[
	style=phys, articletitle=false, biblabel=brackets, chaptertitle=false, pageranges=false, url=true
]{biblatex}
\addbibresource{henkel_notes.bib}

% Document
\begin{document}

	% \graphicspath{{figures/}}

	\section{Henkel notes} \label{sec:NoiseCalc}
	An excellent review of the calculation of relaxation rates is available in~\cite{Henkel1999}.
	The theory behind the noise calculations comes in large part from~\cite{Agarwal1975}.

	\subsection{Noise calc results} \label{subsec:NoiseCalcResults}
	In~\cite{Henkel1999}, the authors find that the noise depends on the spectral field density.
	They consider a system with an infinite conducting plane a distance $z$ away from a trap;
	the geometry here affects the Green functions that they use, so these results are specific to that geometry.
	Starting with the spectral density of a blackbody field, $S^{bb}_E(\omega)$, they have
	\begin{equation}
		S^{(bb)}_E(\omega) = \frac{\hbar \omega^3}{3 \pi \epsilon_0 c^3 (1 - e^{-\flatfrac{\hbar \omega}{T}})}
	\end{equation}
	They then relate this to the near field noise expression:
	\begin{equation}
		S^{(nf)ij}_E(z, \omega) = S^{(bb)}_E(\omega) g_{ij}(\lambda),
	\end{equation}
	where $\lambda = \frac{\omega z}{c}$ and $g_{ij}$ is a dimensionless noise tensor with components in the directions parallel ($g_{xx}$ and $g_{yy}$) and perpendicular ($g_{zz}$) to the surface as follows:
	\begin{gather}
		g_\parallel = \frac34 \Re \int_0^{+\infty} \dd{u} \frac{u}{v} e^{2i \lambda v} \left(r_s(u) + \left(u^2 - 1\right) r_p(u) \right) \\
		g_\perp = \frac32 \Re \int_0^{+\infty} \dd{u} \frac{u^3}{v} e^{2 i \lambda v} r_p(u)
	\end{gather}
	Here $v = \sqrt{1 - u^2}$, and we take the root $v = i \sqrt{u^2 - 1}$ for $u \geq 1$.
	This notation seems preferable over incorporating factors of $\frac{\omega}{c}$, because we can keep everything dimensionless, and it emphasises that the only dependence on $z$ or $\omega$ is via $\lambda$.

	The magnetic noise tensors $h_\parallel$ and $h_\perp$ are exactly the same, except swapping $r_s \leftrightarrow r_p$ and with an extra factor of $\frac{1}{c^2}$:
	\begin{gather}
		S^{(nf)ij}_B(z, \omega) = \frac{S^{(bb)}_E(\omega)}{c^2} h_{ij}(\lambda) \\
		h_\parallel = \frac34 \Re \int_0^{+\infty} \dd{u} \frac{u}{v} e^{2i \lambda v} \left(r_p(u) + \left(u^2 - 1\right) r_s(u) \right) \\
		h_\perp = \frac32 \Re \int_0^{+\infty} \dd{u} \frac{u^3}{v} e^{2 i \lambda v} r_s(u)
	\end{gather}

	\subsection{Reflection coefficients} \label{subsec:ReflectionCoeffs}
	This shows that the only dependence on the properties of the conducting material comes in through the reflection coefficients $r_p$ and $r_s$.
	The standard Fresnel coefficients defined by
	\begin{align}
		r_p(u) = \frac{\epsilon v - \sqrt{\epsilon - u^2}}{\epsilon v + \sqrt{\epsilon - u^2}} \\
		r_s(u) = \frac{v - \sqrt{\epsilon - u^2}}{v + \sqrt{\epsilon - u^2}}
	\end{align}
	(with relative dielectric constant $\epsilon$) can serve as a local equation\todo{add plots showing what the actual graphs of the noise look like with the Fresnel reflection coefficients};
	they can, however, be extended by calculating the surface impedances as described by~\cite{Ford1984}.
	They have, for cases where the initial medium has a response with $\epsilon = 1$,
	\begin{align}
		r_p(u) &= \frac{\pi v - \zeta_p(u)}{\pi v + \zeta_p(u)} \\
		r_s(u) &= \frac{\zeta_s(u) - \frac{\pi}{v}}{\zeta_s(u) + \frac{\pi}{v}} \\
		\zeta_p(u) &= 2i \int_0^\infty \dd{y} \frac{1}{\kappa^2} \left( \frac{y^2}{\epsilon_t(\frac{\omega}{c}\kappa, \omega) - \kappa^2} + \frac{u^2}{\epsilon_\ell(\frac{\omega}{c}\kappa, \omega)} \right) \\
		\zeta_s(u) &= 2i \int_0^\infty \dd{y} \frac{1}{\epsilon_t(\frac{\omega}{c}\kappa, \omega) - \kappa^2} \\
		\kappa^2 &= u^2 + y^2
	\end{align}
	To calculate this for SCs, we can calculate the conductivity $\sigma$ and use (noting that the arguments of these functions are no longer dimensionless) $\epsilon(k, \omega) = 1 + i \frac{4 \pi}{\omega} \sigma(k, \omega)$\todo{is this actually true in this context, or do we need to show that it's actually true in the case here for SCs where we have a nonlocal response?}.

	\newpage
	\listoftodos
	\newpage
	\printbibliography

\end{document}
