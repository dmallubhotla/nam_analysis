% Preamble
\documentclass[11pt]{article}

% Packages
\usepackage{amsmath}
\usepackage{amssymb}
\usepackage{siunitx}
\usepackage{physics}
%\usepackage{todonotes}
\usepackage{graphicx}
\usepackage{url}
\usepackage[plain]{fancyref}
\usepackage[
style=phys, articletitle=false, biblabel=brackets, chaptertitle=false, pageranges=false, url=true
]{biblatex}
\addbibresource{sc_normal_compare.bib}

% Document
% !TeX spellcheck = en_GB
\begin{document}

	\graphicspath{{figures/}}

	\section{Comparing SC noise to N noise} \label{sec:intro}

	Comparing the Lindhard function results from~\cite{SolyomV3} to what we get if we use the results in~\cite{Nam1967}.

	\begin{figure}[htp]
		\centering
		\includegraphics[width=12cm]{T1ZE1}
		\caption{$T_{1}^{(E)}(z)$, with $z$ in units of $\flatfrac{\omega}{c}$ and $T_1$ in seconds.
		Temperature is $T = .8 T_c$}\label{fig:t1ez}
	\end{figure}

	\begin{figure}[htp]
		\centering
		\includegraphics[width=12cm]{T1ZE3}
		\caption{$T_{1}^{(E)}(z)$, with $z$ in units of $\flatfrac{\omega}{c}$ and $T_1$ in seconds.
		Temperature is $T = .95 T_c$}\label{fig:t1ezpart2}
	\end{figure}

	\begin{figure}[htp]
		\centering
		\includegraphics[width=12cm]{T1ZE4}
		\caption{$T_{1}^{(E)}(z)$, with $z$ in units of $\flatfrac{\omega}{c}$ and $T_1$ in seconds.
		Temperature is $T = .995 T_c$}\label{fig:t1ezpart3}
	\end{figure}


	The three different curves labelled ``Nam'' in \fref{fig:t1ez} represent three different values for the cutoff imposed to prevent divergence.
	Respectively, for curves $(1, 2, 3)$ they are $u_{uc} = 10^{(9, 10, 11)}$.
	In \fref{fig:t1ezpart2} and \fref{fig:t1ezpart3}, we see the same functions plotted at higher temperatures.
	For the normal state, the only scale for temperature is $\flatfrac{\omega}{T}$, so the normal state does not change, whereas for the superconducting case we see that higher temperatures correspond with more fluctuations and a shorter $T_1$.

%	\newpage
%	\listoftodos
	\newpage
	\printbibliography

\end{document}
