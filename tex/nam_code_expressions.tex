\providecommand{\main}{..}
\documentclass[../main.tex]{subfiles}

\title{Nam Code Expressions}
\author{\begin{telugu}హృదయ్ దీపక్ మల్లుభొట్ల\end{telugu}}
% want empty date
\predate{}
\date{}
\postdate{}

% Document
% !TeX spellcheck = en_GB
\begin{document}

	\onlyinsubfile{\maketitle}
	\onlyinsubfile{\tableofcontents}

	\section{Nam Code} \label{sec:nce}

	\subsection{$\epsilon_{lindhard}$} \label{subsec:nce:epsl}

	These are the expressions from the Lindhard dielectric function, as in Solyom\supercite{SolyomV3}.
	Here, the parameters are in SI units.
	Collision frequency $\nu$.
	Exception is the momentum parameter $u$, which is in terms of inverse vacuum wavelength.
	This is become it comes from the noise calculation.

	Define the following helper variables:
	\begin{gather}
		s = \flatfrac{\nu}{\omega} \\
		\kappa = u \frac{\vf}{c}
	\end{gather}

	Now,
	\begin{gather}
		\epsilon_{lindhard} = 1 + \frac{3 \omega_p^2}{\omega^2} \frac{1}{\kappa^2} \frac{1 +\frac{1 + is}{2 \kappa} \log \frac{1 - \kappa + i s}{1 + \kappa + i s}}{1 + \frac{is}{2 \kappa} \log \frac{1 - \kappa + i s}{1 + \kappa + i s}} \label{eq:nce:epsLinFull}
	\end{gather}

	\subsubsection{Lindhard small momentum series}

	For $\kappa \gg 1$,~\eqref{eq:nce:epsLinFull} is fine.
	In the region where $\kappa \ll 1$, the computer does not handle the logs effectively.
	Instead, using a series expansion is preferable:
	\begin{gather}
		\epsilon_{lindhard, series} = 1 + \frac{3 \omega_p^2}{\omega^2} \left(\frac{i}{3 (s - i)} + u^2  \frac{-9i + 5 s}{45 (s - i)^3} \right)
	\end{gather}

	\subsection{$\epsilon_{nam}$} \label{subsec:nce:epsn}

	We get the Nam dielectric function from~\cite{Nam1967}.
	Removing the dimensions from the physical parameters and constants gives:
	\begin{align}
		\xi &= \frac{\omega}{\Delta} \\
		\xi' &= \frac{\omega'}{\Delta} \\
		\nu &= \frac{1}{\tau \Delta} \\
		\kappa &= \frac{q \vf}{\Delta} \\
		t &= \frac{T}{\Delta} \\
		\sigma_0 &= \frac{n e^2}{m \Delta}
	\end{align}

	From Nam, we have
	\begin{equation}
		\sigma(\kappa, \xi) = -i \frac{3 \sigma_0}{4} \frac{1}{\xi}\left[\int_{1 - \xi}^{1}\dd{\xi} \tanh(\frac{\xi + \xi'}{2 t}) I_1 + \int_{1}^{\infty} \dd{\xi'} \left( \tanh(\frac{\xi + \xi'}{2t}) I_1  - \tanh(\frac{\xi'}{2t})I_2 \right) \right]
	\end{equation}
	with
	\begin{align}
		I_1 &= F(\kappa, \Re[\sqrt{(\xi + \xi')^2 - 1} - \sqrt{\xi'^2 - 1}]) (g + 1) \nonumber\\
		&\quad + F(\kappa, \Re[-\sqrt{(\xi + \xi')^2 - 1} - \sqrt{\xi'^2 - 1}]) (g - 1) \\
		I_2 &= F(\kappa, \Re[\sqrt{(\xi + \xi')^2 - 1} - \sqrt{\xi'^2 - 1}]) (g + 1) \nonumber\\
		&\quad + F(\kappa, \Re[\sqrt{(\xi + \xi')^2 - 1} + \sqrt{\xi'^2 - 1}]) (g - 1) \\
		F(\kappa, E) &= \frac{1}{\kappa} \left[2 S(E) + (1 - S(E)^2)\ln(\frac{S(E) + 1}{S(E) - 1})\right]  \\
		S(\kappa, E) &= \frac{1}{\kappa} \left(E - i \left(\Im[\sqrt{(\xi + \xi')^2 - 1} + \sqrt{\xi'^2 - 1}] + 2 \nu \right) \right) \\
		g  &= \frac{\xi' \left( \xi + \xi'\right) + 1}{\sqrt{\xi'^2 - 1}\sqrt{(\xi + \xi')^2 - 1}}
	\end{align}

	Then,
	\begin{gather}
		\epsilon = 1 + i \frac{4 \pi}{\omega} \sigma
	\end{gather}

	\subsubsection{Nam low momentum}
	As with the Lindhard case, we want a separate expression that holds for small momenta.
	Upon series expanding, we get
	\begin{align}
		F =\frac43 \frac{1}{\eta} + \frac{4}{15} \frac{1}{\eta^3} \kappa^2,
	\end{align}
	where
	\begin{equation}
		\eta = E - i \left(\Im[\sqrt{(\xi + \xi')^2 - 1} + \sqrt{\xi'^2 - 1}] + 2 \nu \right).
	\end{equation}

	Substituting this expression for $F$ back in the general Nam expression gives us our low $\kappa$ form.

	\subsubsection{Nam simple dielectric approximated form}

	On its own, this is an expensive calculation.
	However, we can estimate this using a piecewise description of $\epsilon_{nam}$ that approximates the full expression.

	Quoting from an earlier document, the procedure to obtain this is as follows.
	\begin{enumerate}
		\item Generally, $\epsilon(u) = 1 + 4 \pi i \frac{\sigma_N}{\omega} \Sigma(\xi, A u, \tau, t)$.
		Use the asymptotic expression for $\Sigma$ as $u \rightarrow 0$.
		Then,
		\begin{equation}
			-a + ib = 4 \pi i \frac{\sigma_N}{\omega} \Sigma(A u \rightarrow 0),
		\end{equation}
		from which $a$ and $b$ can be extracted by taking the real and imaginary parts.
		\item Similarly, for large $\kappa = A u$,
		\begin{align}
			\frac{-c + i d}{u} &= 4 \pi i \frac{\sigma_N}{\omega} \Sigma(\kappa > u_c) \\
			- c + i d &= 4 \pi i \frac{\sigma_N}{\omega} \Sigma(\kappa > u_c) u \\
			- c + i d &= 4 \pi i \frac{\sigma_N}{\omega} \Sigma(\kappa > u_c) \frac{\kappa}{A}
		\end{align}
		If we want to determine $c$ and $d$, we can simply increase $\kappa$ until this expression stops varying greatly, which in practice will be determined by $\tau$ and $\vf$ (as should be expected).
		\item We can determine an interpolation constant by finding $u_c$ such that
		\begin{align}
			-a + ib &= 1 + \frac{-c + id}{u_c} \\
			u_c &= \frac{-c + id}{-(a + 1) + ib}.
		\end{align}
		\item With $a$, $b$, $c$, $d$, and $u_c$, we now have enough to define our asymptotic expression
		\begin{equation}
			\epsilon(u, u_{uc}) =
			\begin{cases}
				-a + i b & u < u_c \\
				1 + \frac{-c + i d}{u} &  u_{uc} > u \geq u_c \\
				1 & u > u_{uc}
			\end{cases}.
		\end{equation}

	\end{enumerate}

	In all of the previous, we end up with a function $\epsilon(u)$, with $u$ in units of inverse vacuum wavelength.
	For all cases, the dielectric function is all we need to characterise the physical situation, which is why it's specifically important to trace out all of the parameters starting from SI units.

	\subsection{Surface Impedance} \label{subsec:nce:surfaceimpedance}
	Now, we use expressions from~\cite{Ford1984} to calculate the surface impedance using the dielectric function.

	\subsubsection{$\zeta_p$}
	We start with $\zeta_p$.
	\begin{equation}
		\zeta_p = 2i \int_0^\infty \dd{y} \frac{1}{u^2 + y^2} \left( \frac{y^2}{\epsilon(\sqrt{u^2 + y^2}, \omega) - u^2 - y^2} + \frac{u^2}{\epsilon(\sqrt{u^2 + y^2}, \omega)} \right)
	\end{equation}

	This isn't on its own inherently easy to calculate numerically, so we can try to mitigate that by rescaling $y$.
	Defining $x = \flatfrac{y}{u}$,
	\begin{align}
		\zeta_p &= 2i \int_0^\infty u \dd{x} \frac{1}{u^2 + (u x)^2} \left( \frac{(u x)^2}{\epsilon(\sqrt{u^2 + (u x)^2}, \omega) - u^2 - (u x)^2} + \frac{u^2}{\epsilon(\sqrt{u^2 + (u x)^2}, \omega)} \right) \\
		\zeta_p &= 2i u \int_0^\infty \dd{x} \frac{1}{u^2} \frac{1}{1 + x^2} \left( \frac{u^2 x^2}{\epsilon(u \sqrt{1 + x^2}, \omega) - u^2 - u^2 x^2} + \frac{u^2}{\epsilon(u \sqrt{1 + x^2}, \omega)} \right) \\
		\zeta_p &= 2i u \int_0^\infty \dd{x} \frac{1}{1 + x^2} \left( \frac{x^2}{\epsilon(u \sqrt{1 + x^2}, \omega) - u^2 - u^2 x^2} + \frac{1}{\epsilon(u \sqrt{1 + x^2}, \omega)} \right)
	\end{align}

	\subsection{Reflection Coefficients} \label{subsec:nce:reflectcoefs}
	\subsubsection{Imaginary part of $r_p$}
	The imaginary part can be calculated from $\zeta_p$.

\end{document}
