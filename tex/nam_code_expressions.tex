\providecommand{\main}{..}
\documentclass[../main.tex]{subfiles}

\title{Nam Code Expressions}
\author{}
% want empty date
\predate{}
\date{}
\postdate{}

% Document
% !TeX spellcheck = en_GB
\begin{document}

	\onlyinsubfile{\maketitle}
	\onlyinsubfile{\tableofcontents}

	\section{Nam Code} \label{sec:nce}

	\subsection{$\epsilon_{lindhard}$} \label{subsec:nce:epsl}

	These are the expressions from the Lindhard dielectric function, as in Solyom\supercite{SolyomV3}.
	Here, the parameters are in SI units.
	Collision frequency $\nu$.
	Exception is the momentum parameter $u$, which is in terms of inverse vacuum wavelength.
	This is become it comes from the noise calculation.

	Define the following helper variables:
	\begin{gather}
		s = \flatfrac{\nu}{\omega} \\
		\kappa = u \frac{\vf}{c}
	\end{gather}

	Now,
	\begin{gather}
		\epsilon_{lindhard} = 1 + \frac{3 \omega_p^2}{\omega^2} \frac{1}{\kappa^2} \frac{1 +\frac{1 + is}{2 \kappa} \log \frac{1 - \kappa + i s}{1 + \kappa + i s}}{1 + \frac{is}{2 \kappa} \log \frac{1 - \kappa + i s}{1 + \kappa + i s}} \label{eq:nce:epsLinFull}
	\end{gather}

	\subsubsection{Lindhard small momentum series}

	For $\kappa \gg 1$,~\eqref{eq:nce:epsLinFull} is fine.
	In the region where $\kappa \ll 1$, the computer does not handle the logs effectively.
	Instead, using a series expansion is preferable:
	\begin{gather}
		\epsilon_{lindhard, series} = 1 + \frac{3 \omega_p^2}{\omega^2} \left(\frac{i}{3 (s - i)} + u^2  \frac{-9i + 5 s}{45 (s - i)^3} \right)
	\end{gather}

	\subsection{$\epsilon_{nam}$} \label{subsec:nce:epsn}

	We get the Nam dielectric function from~\cite{Nam1967}.
	Removing the dimensions from the physical parameters and constants gives:
	\begin{align}
		\xi &= \frac{\omega}{\Delta} \\
		\xi' &= \frac{\omega'}{\Delta} \\
		\nu &= \frac{1}{\tau \Delta} \\
		\kappa &= \frac{q \vf}{\Delta} \\
		t &= \frac{T}{\Delta} \\
		\sigma_0 &= \frac{n e^2}{m \Delta}
	\end{align}

	From Nam, we have
	\begin{equation}
		\sigma(\kappa, \xi) = -i \frac{3 \sigma_0}{4} \frac{1}{\xi}\left[\int_{1 - \xi}^{1}\dd{\xi} \tanh(\frac{\xi + \xi'}{2 t}) I_1 + \int_{1}^{\infty} \dd{\xi'} \left( \tanh(\frac{\xi + \xi'}{2t}) I_1  - \tanh(\frac{\xi'}{2t})I_2 \right) \right]
	\end{equation}
	with
	\begin{align}
		I_1 &= F(\kappa, \Re[\sqrt{(\xi + \xi')^2 - 1} - \sqrt{\xi'^2 - 1}]) (g + 1) \nonumber\\
		&\quad + F(\kappa, \Re[-\sqrt{(\xi + \xi')^2 - 1} - \sqrt{\xi'^2 - 1}]) (g - 1) \\
		I_2 &= F(\kappa, \Re[\sqrt{(\xi + \xi')^2 - 1} - \sqrt{\xi'^2 - 1}]) (g + 1) \nonumber\\
		&\quad + F(\kappa, \Re[\sqrt{(\xi + \xi')^2 - 1} + \sqrt{\xi'^2 - 1}]) (g - 1) \\
		F(\kappa, E) &= \frac{1}{\kappa} \left[2 S(E) + (1 - S(E)^2)\ln(\frac{S(E) + 1}{S(E) - 1})\right]  \\
		S(\kappa, E) &= \frac{1}{\kappa} \left(E - i \left(\Im[\sqrt{(\xi + \xi')^2 - 1} + \sqrt{\xi'^2 - 1}] + 2 \nu \right) \right) \\
		g  &= \frac{\xi' \left( \xi + \xi'\right) + 1}{\sqrt{\xi'^2 - 1}\sqrt{(\xi + \xi')^2 - 1}}
	\end{align}

	Then,
	\begin{gather}
		\epsilon = 1 + i \frac{4 \pi}{\omega} \sigma
	\end{gather}

	\subsubsection{Nam low momentum}
	As with the Lindhard case, we want a separate expression that holds for small momenta.
	Upon series expanding, we get
	\begin{align}
		F =\frac43 \frac{1}{\eta} + \frac{4}{15} \frac{1}{\eta^3} \kappa^2,
	\end{align}
	where
	\begin{equation}
		\eta = E - i \left(\Im[\sqrt{(\xi + \xi')^2 - 1} + \sqrt{\xi'^2 - 1}] + 2 \nu \right).
	\end{equation}

	Substituting this expression for $F$ back in the general Nam expression gives us our low $\kappa$ form.

	\subsubsection{Nam simple dielectric approximated form}

	On its own, this is an expensive calculation.
	However, we can estimate this using a piecewise description of $\epsilon_{nam}$ that approximates the full expression.

	Quoting from an earlier document, the procedure to obtain this is as follows.\footnote{This is the procedure that made it possible to calculate the surface impedances in a reasonable amount of time.
	It looks right graphically, and relies on the fact that all that matters is the correctness of $\epsilon$ in a narrow range.
	I still don't have a great justification for why it should have a small error bound.}
	\begin{enumerate}
		\item Generally, $\epsilon(u) = 1 + 4 \pi i \frac{\sigma_N}{\omega} \Sigma(\xi, A u, \tau, t)$.
		Use the asymptotic expression for $\Sigma$ as $u \rightarrow 0$.
		Then,
		\begin{equation}
			-a + ib = 4 \pi i \frac{\sigma_N}{\omega} \Sigma(A u \rightarrow 0),
		\end{equation}
		from which $a$ and $b$ can be extracted by taking the real and imaginary parts.
		\item Similarly, for large $\kappa = A u$,
		\begin{align}
			\frac{-c + i d}{u} &= 4 \pi i \frac{\sigma_N}{\omega} \Sigma(\kappa > u_c) \\
			- c + i d &= 4 \pi i \frac{\sigma_N}{\omega} \Sigma(\kappa > u_c) u \\
			- c + i d &= 4 \pi i \frac{\sigma_N}{\omega} \Sigma(\kappa > u_c) \frac{\kappa}{A}
		\end{align}
		If we want to determine $c$ and $d$, we can simply increase $\kappa$ until this expression stops varying greatly, which in practice will be determined by $\tau$ and $\vf$ (as should be expected).
		\item We can determine an interpolation constant by finding $u_c$ such that
		\begin{align}
			-a + ib &= 1 + \frac{-c + id}{u_c} \\
			u_c &= \frac{-c + id}{-(a + 1) + ib}.
		\end{align}
		\item With $a$, $b$, $c$, $d$, and $u_c$, we now have enough to define our asymptotic expression
		\begin{equation}
			\epsilon(u, u_{uc}) =
			\begin{cases}
				-a + i b & u < u_c \\
				1 + \frac{-c + i d}{u} &  u_{uc} > u \geq u_c \\
				1 & u > u_{uc}
			\end{cases}.
		\end{equation}
		The parameter $u_{uc}$ is the upper cutoff, which is necessary for the asymptotic $\frac{1}{u}$ behaviour of the Nam dielectric function to not diverge later.\footnote{
			This is where the ``cutoff'' gets inserted.
			Based on our discussions, this should be the expected way to implement it.
		}

	\end{enumerate}

	In all of the previous, we end up with a function $\epsilon(u)$, with $u$ in units of inverse vacuum wavelength.
	For all cases, the dielectric function is all we need to characterise the physical situation, which is why it's specifically important to trace out all of the parameters starting from SI units.

	\subsection{Surface Impedance} \label{subsec:nce:surfaceimpedance}
	Now, we use expressions from~\cite{Ford1984} to calculate the surface impedance using the dielectric function.

	\subsubsection{$\zeta_p$}
	We start with $\zeta_p$.
	\begin{equation}
		\zeta_p(u) = 2i \int_0^\infty \dd{y} \frac{1}{u^2 + y^2} \left( \frac{y^2}{\epsilon(\sqrt{u^2 + y^2}, \omega) - u^2 - y^2} + \frac{u^2}{\epsilon(\sqrt{u^2 + y^2}, \omega)} \right)
	\end{equation}

	This isn't on its own inherently easy to calculate numerically, so we can try to mitigate that by rescaling $y$.
	Defining $x = \flatfrac{y}{u}$,
	\begin{align}
		\zeta_p &= 2i \int_0^\infty u \dd{x} \frac{1}{u^2 + (u x)^2} \left( \frac{(u x)^2}{\epsilon(\sqrt{u^2 + (u x)^2}, \omega) - u^2 - (u x)^2} + \frac{u^2}{\epsilon(\sqrt{u^2 + (u x)^2}, \omega)} \right) \\
		\zeta_p &= 2i u \int_0^\infty \dd{x} \frac{1}{u^2} \frac{1}{1 + x^2} \left( \frac{u^2 x^2}{\epsilon(u \sqrt{1 + x^2}, \omega) - u^2 - u^2 x^2} + \frac{u^2}{\epsilon(u \sqrt{1 + x^2}, \omega)} \right) \\
		\zeta_p &= 2i u \int_0^\infty \dd{x} \frac{1}{1 + x^2} \left( \frac{x^2}{\epsilon(u \sqrt{1 + x^2}, \omega) - u^2 - u^2 x^2} + \frac{1}{\epsilon(u \sqrt{1 + x^2}, \omega)} \right)
	\end{align}

	The interesting features of this integral will be around $x \approx 1$, which is what we need for the computer to handle the integral to $\infty$.

	\subsubsection{$\zeta_s$}

	The $\zeta_s$ form is simpler.

	\begin{align}
		\zeta_s(u) &= 2i \int_0^\infty \dd{y} \frac{1}{\epsilon_t(\sqrt{u^2 + y^2}, \omega) - u^2 - y^2}
	\end{align}

	This can again be simplified to give an integral with a peak around one.

	\subsection{Reflection Coefficients} \label{subsec:nce:reflectcoefs}
	\subsubsection{Imaginary part of $r$}

	The reflection coefficient is a simple function of the surface impedance, again from~\cite{Ford1984}.
	These expressions will be integrated in an expression with peak where $u \gg 1$.
	In Ford and Weber, as well as in the expressions for the noise, other authors use $v = \sqrt{1 - u^2}$, but it makes the expressions more amenable to analysis to write $v = i u$.
	Writing it here also enforces the necessary branch of the square root function.
	When we calculate the noise in a moment, in the same approximation we will only need the imaginary part of the reflection coefficient.

	\begin{align}
		\Im r_p(u) &= \Im \frac{i \pi u - \zeta_p(u)}{i \pi u + \zeta_p(u)} \\
		\Im r_s(u) &= \Im \frac{\zeta_s(u) - \frac{\pi}{i u}}{\zeta_s(u) + \frac{\pi}{i u}}
	\end{align}

	\subsubsection{Interpolation}

	As mentioned above, the Nam dielectric function has unrealistic behaviour at large momentum.
	The $\frac{1}{u}$ behaviour makes the noise diverge, rather than showing the nonlocal flattening we want to see.
	Using the cutoff parameter earlier is arbitrary, and in the range of values that make physical sense the result is strongly dependent on the value.

	To combat this, an alternative method is to interpolate the Nam and Lindhard reflection coefficient expressions.
	This has the advantage of ensuring that both have the same asymptotic behaviour as $T \rightarrow T_c$ (which is a good smoke test that this isn't obviously incorrect).

	We then define
	\begin{equation}
		\Im r_{interpolated, (r, s)} = \min(\Im r_{lindhard, (r, s)}, \Im r_{nam, (r, s)})
	\end{equation}

	This in theory is a function of $u$ that gives a better approximation to the imaginary part of $r_p$ or $r_s$ for BCS-type metals, where the Lindhard form would fail.
	This also does not fail for large $u$ in the same way as the Nam form.

	It seems intuitively plausible that this expression works for small $u$ where the Nam form will be valid, as well as in the very large $u$ limit ($u \gg u_F$).
	The risk comes in the region where $u \approx u_F$, where there is presumably a more intricate description required.
	However, this function gets sampled by our noise integral around $u \sim \frac{1}{z}$, so for $z$ not around the inverse Fermi momentum, we might feel more confident.

	The cutoff can still play a role, but this interpolation procedure takes away the very sensitive dependence on $u_{uc}$, so in the code we can simply pick $u_{uc} = u_F$.

	\subsection{Noise calculation} \label{subsec:nce:noise}

	Now, from~\cite{Henkel1999} or~\cite{QubitRelax}, we have the following expressions for the noise:

	\begin{align}
		S^{(nf)ij}_E(z, \omega) &= S^{(bb)}_E(\omega) g_{ij}(z) \\
		S^{(bb)}_E(\omega) &= \frac{\hbar \omega^3}{3 \pi \epsilon_0 c^3 (1 - e^{-\flatfrac{\hbar \omega}{T}})}
	\end{align}

	where $z$ is in units of the vacuum wavelength $\frac{c}{\omega}$ and $g_{ij}$ is a dimensionless noise tensor with components in the directions parallel ($g_{xx}$ and $g_{yy}$) and perpendicular ($g_{zz}$) to the surface as follows:

	\begin{gather}
		g_\parallel = \frac34 \Re \int_0^{+\infty} \dd{u} \frac{u}{v} e^{2i z v} \left(r_s(u) + \left(u^2 - 1\right) r_p(u) \right) \\
		g_\perp = \frac32 \Re \int_0^{+\infty} \dd{u} \frac{u^3}{v} e^{2 i z v} r_p(u)
	\end{gather}

	The magnetic noise tensors $h_\parallel$ and $h_\perp$ are exactly the same, except swapping $r_s \leftrightarrow r_p$ and with an extra factor of $\frac{1}{c^2}$:

	\begin{gather}
		S^{(nf)ij}_B(z, \omega) = \frac{S^{(bb)}_E(\omega)}{c^2} h_{ij}(z) \\
		h_\parallel = \frac34 \Re \int_0^{+\infty} \dd{u} \frac{u}{v} e^{2i z v} \left(r_p(u) + \left(u^2 - 1\right) r_s(u) \right) \\
		h_\perp = \frac32 \Re \int_0^{+\infty} \dd{u} \frac{u^3}{v} e^{2 i z v} r_s(u)
	\end{gather}

	Now using the replacement $v \rightarrow iu$, we get

	\begin{align}
		g_\parallel &= \frac34 \int_0^{+\infty} \dd{u} e^{- 2 z u} \left(\Im r_s(u) + \left(u^2 - 1\right) \Im r_p(u) \right) \\
		g_\perp &= \frac32 \int_0^{+\infty} \dd{u} u^2 e^{- 2 z u} \Im r_p(u) \\
		h_\parallel &= \frac34 \int_0^{+\infty} \dd{u} e^{- 2 z u} \left(\Im r_p(u) + \left(u^2 - 1\right) \Im r_s(u) \right) \\
		h_\perp &= \frac32 \int_0^{+\infty} \dd{u} u^2 e^{-2 z u} \Im r_s(u)
	\end{align}

	Obviously, the lower limit of this integral is in a region where $v \not\approx iu$, but we can rely on the fact that in the perpendicular expressions the $u^2$ factor damps out any inaccuracies.

\end{document}
