\documentclass[../main.tex]{subfiles}

\title{Magnetic noise interpolation}
\author{\begin{telugu}హృదయ్ దీపక్ మల్లుభొట్ల\end{telugu}}
% want empty date
\predate{}
\date{}
\postdate{}

% Document
% !TeX spellcheck = en_GB
\begin{document}



	\onlyinsubfile{\maketitle}

	\section{Magnetic noise} \label{sec:imn:intro}

	Calculating $\chi_{BB}(z)$, via
	\begin{align}
		\chi_{BB} \propto \Re \int_0^{+\infty} \dd{u} \frac{u^3}{v} e^{2 i z v} r_s(u),
	\end{align}
	where
	\begin{align}
		r_s(u) &= \frac{\zeta_s(u) - \frac{\pi}{v}}{\zeta_s(u) + \frac{\pi}{v}} \\
		\zeta_s(u) &= 2i \int_0^\infty \dd{y} \frac{1}{\epsilon_t(\frac{\omega}{c}\kappa, \omega) - \kappa^2} \\
		\kappa^2 &= u^2 + y^2
	\end{align}
	Again, here using interpolated Nam form approximating the expression in Nam\supercite{Nam1967}.
	In \fref{fig:cutoff} we can see how the cutoff parameter affects the magnetic noise.
	We see that unlike the electric case, the cutoff only matters at distances far smaller than the relevant distance scale, which squares with the idea that the magnetic noise expressions converge better.
	\begin{figure}[htp]
		\centering
		\includegraphics[width=12cm]{chiZBarounduFVaryingCutoff}
		\caption{$\chi_{BB}(z)$, with $z$ in units of $\flatfrac{\omega}{c}$, for different $u_{uc}$} \label{fig:cutoff}
	\end{figure}

	\subsection{Magnetic noise convergence} \label{subsec:magneticnoiseconvergence}

	We can show that
	\begin{equation}
		\zeta_s(u) \sim - \frac{i \pi u}{2 \sqrt{u\left( - c + di + u - u^3 \right)}},
	\end{equation}
	which implies that for sufficiently large $u$,
	\begin{align}
		\zeta_s(u) &\sim - \frac{i \pi u}{2 \sqrt{- u^4}} \\
		&\sim - \frac{i \pi u}{2 i u^2} \\
		&\sim - \frac{\pi }{2 u}
	\end{align}
	Some more explicit series expansion\todo{Finish typing this up and prove this} shows that the imaginary part goes as $\frac{1}{u^2}$
	So asymptotically $\zeta_s \sim \flatfrac{A}{u} + \flatfrac{i B}{u^2}$, where $A, B$ carry sign depending on the sign chosen in the square root.
	We can insert that into $r_s$ to see what happens there:
	\begin{align}
		r_s(u) &= \frac{\zeta_s(u) - \frac{\pi}{iu }}{\zeta_s(u) + \frac{\pi}{iu}} \\
		r_s(u) &= \frac{\flatfrac{A}{u} + \flatfrac{i B}{u^2} - \frac{\pi}{iu}}{\flatfrac{A}{u} + \flatfrac{i B}{u^2} + \frac{\pi}{iu}} \\
		r_s(u) &= \frac{A u + i B - \pi u}{A u + i B + \pi u} \\
		r_s(u) &= \frac{(A - \pi) u + i B}{(A + \pi)u + i B } \\
		r_s(u) &= \frac{(A - \pi) u + i B}{(A + \pi)u + i B } \frac{{(A + \pi)u - i B }}{{(A + \pi)u - i B }} \\
		r_s(u) &= \frac{(A^2 - \pi^2) u^2 + B^2  + i B u(A + \pi - A + \pi)}{(A + \pi)^2 u^2 + B^2 }
	\end{align}
	So the imaginary part of $r_s$ still goes at $\frac{1}{u}$.\todo{Verify this, this seems off.}


	\section{Magnetic noise plots} \label{sec:ime:plots}

	Figures~\ref{fig:ime:temp},~\ref{fig:ime:frequency},~\ref{fig:ime:tau} and~\ref{fig:ime:vf} are plotted for a cutoff of $u_{uc} = 10^{10}$.

	\begin{figure}[htp]
		\centering
		\includegraphics[width=12cm]{chiZBarounduFVaryingTemp}
		\caption{$\chi_{BB}(z)$, with $z$ in units of $\flatfrac{\omega}{c}$, for different $T$} \label{fig:ime:temp}
	\end{figure}

	\begin{figure}[htp]
		\centering
		\includegraphics[width=12cm]{chiZBarounduFVaryingFrequency}
		\caption{$\chi_{BB}(z)$, with $z$ in units of $\flatfrac{\omega}{c}$, for different frequency} \label{fig:ime:frequency}
	\end{figure}

	\begin{figure}[htp]
		\centering
		\includegraphics[width=12cm]{chiZBarounduFVaryingTau}
		\caption{$\chi_{BB}(z)$, with $z$ in units of $\flatfrac{\omega}{c}$, for different $\tau$} \label{fig:ime:tau}
	\end{figure}

	\begin{figure}[htp]
		\centering
		\includegraphics[width=12cm]{chiZBarounduFVaryingVf}
		\caption{$\chi_{BB}(z)$, with $z$ in units of $\flatfrac{\omega}{c}$, for different $\vf$} \label{fig:ime:vf}
	\end{figure}


\end{document}
