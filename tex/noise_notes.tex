\providecommand{\main}{..}
\documentclass[../main.tex]{subfiles}

\title{Noise discussion}
\author{\begin{telugu}హృదయ్ దీపక్ మల్లుభొట్ల\end{telugu}}
% want empty date
\predate{}
\date{}
\postdate{}

% Document
% !TeX spellcheck = en_GB
\begin{document}

	\graphicspath{{\main/figures/}}

	\onlyinsubfile{\maketitle}
	\section{Noise notes} \label{sec:NoiseCalc}
	An excellent review of the calculation of relaxation rates is available in~\cite{Henkel1999}.
	The theory behind the noise calculations comes in large part from~\cite{Agarwal1975}.

	\subsection{Noise calc results} \label{subsec:NoiseCalcResults}
	In~\cite{Henkel1999}, the authors find that the noise depends on the spectral field density.
	They consider a system with an infinite conducting plane a distance $z$ away from a trap;
	the geometry here affects the Green functions that they use, so these results are specific to that geometry.
	Starting with the spectral density of a blackbody field, $S^{bb}_E(\omega)$, they have
	\begin{equation}
		S^{(bb)}_E(\omega) = \frac{\hbar \omega^3}{3 \pi \epsilon_0 c^3 (1 - e^{-\flatfrac{\hbar \omega}{T}})}
	\end{equation}
	They then relate this to the near field noise expression:
	\begin{equation}
		S^{(nf)ij}_E(z, \omega) = S^{(bb)}_E(\omega) g_{ij}(z),
	\end{equation}
	where $z$ is in units of the vacuum wavelength $\frac{c}{omega}$ and $g_{ij}$ is a dimensionless noise tensor with components in the directions parallel ($g_{xx}$ and $g_{yy}$) and perpendicular ($g_{zz}$) to the surface as follows:
	\begin{gather}
		g_\parallel = \frac34 \Re \int_0^{+\infty} \dd{u} \frac{u}{v} e^{2i z v} \left(r_s(u) + \left(u^2 - 1\right) r_p(u) \right) \\
		g_\perp = \frac32 \Re \int_0^{+\infty} \dd{u} \frac{u^3}{v} e^{2 i z v} r_p(u)
	\end{gather}
	Here $v = \sqrt{1 - u^2}$, and we take the root $v = i \sqrt{u^2 - 1}$ for $u \geq 1$.
	This notation seems preferable over incorporating factors of $\frac{\omega}{c}$, because we can keep everything dimensionless, and it emphasises that the only dependence on $z$ or $\omega$ is via $z$.

	The magnetic noise tensors $h_\parallel$ and $h_\perp$ are exactly the same, except swapping $r_s \leftrightarrow r_p$ and with an extra factor of $\frac{1}{c^2}$:
	\begin{gather}
		S^{(nf)ij}_B(z, \omega) = \frac{S^{(bb)}_E(\omega)}{c^2} h_{ij}(z) \\
		h_\parallel = \frac34 \Re \int_0^{+\infty} \dd{u} \frac{u}{v} e^{2i z v} \left(r_p(u) + \left(u^2 - 1\right) r_s(u) \right) \\
		h_\perp = \frac32 \Re \int_0^{+\infty} \dd{u} \frac{u^3}{v} e^{2 i z v} r_s(u)
	\end{gather}

	\subsection{Reflection coefficients} \label{subsec:ReflectionCoeffs}
	This shows that the only dependence on the properties of the conducting material comes in through the reflection coefficients $r_p$ and $r_s$.
	The standard Fresnel coefficients defined by
	\begin{align}
		r_p(u) = \frac{\epsilon v - \sqrt{\epsilon - u^2}}{\epsilon v + \sqrt{\epsilon - u^2}} \\
		r_s(u) = \frac{v - \sqrt{\epsilon - u^2}}{v + \sqrt{\epsilon - u^2}}
	\end{align}
	(with relative dielectric constant $\epsilon$) can serve as a local equation\todo{add plots showing what the actual graphs of the noise look like with the Fresnel reflection coefficients};
	they can, however, be extended by calculating the surface impedances as described by~\cite{Ford1984} (with similar results appearing in~\cite{Nam1967_Part2}).
	They have, for cases where the initial medium has a response with $\epsilon = 1$\todo{rephrase},
	\begin{align}
		r_p(u) &= \frac{\pi v - \zeta_p(u)}{\pi v + \zeta_p(u)} \\
		r_s(u) &= \frac{\zeta_s(u) - \frac{\pi}{v}}{\zeta_s(u) + \frac{\pi}{v}} \\
		\zeta_p(u) &= 2i \int_0^\infty \dd{y} \frac{1}{\kappa^2} \left( \frac{y^2}{\epsilon_t(\frac{\omega}{c}\kappa, \omega) - \kappa^2} + \frac{u^2}{\epsilon_\ell(\frac{\omega}{c}\kappa, \omega)} \right) \\
		\zeta_s(u) &= 2i \int_0^\infty \dd{y} \frac{1}{\epsilon_t(\frac{\omega}{c}\kappa, \omega) - \kappa^2} \\
		\kappa^2 &= u^2 + y^2
	\end{align}
	To calculate this for SCs, we can calculate the conductivity $\sigma$ and use (noting that the arguments of these functions are no longer dimensionless) $\epsilon(k, \omega) = 1 + i \frac{4 \pi}{\omega} \sigma(k, \omega)$%\todo{is this actually true in this context, or do we need to show that it's actually true in the case here for SCs where we have a nonlocal response?}.

	\section{Sample calculations} \label{sec:SampleCalcs}

	\subsection{Good conductor Fresnel noise} \label{subsec:sample:fresnelperfect}

	As a sanity check we can look at the case of a perfect conductor, with $\epsilon = + i \infty$ (or $\epsilon = 1 + i \infty$, the $1$ doesn't matter).
	This gives us $r_p(u) = 1 + 0 i$ and $r_p(u) = -1 + 0 i$.
	It turns out that the noise integral can be done analytically in that case, giving
	\begin{align}
		g_{\parallel} &= \frac{3}{4} \left(\frac{\sin{2 z}}{4 z^3} - \frac{\cos{2 z}}{2 z^2} - \frac{\sin{2 z}}{z}\right) \\
		g_{\perp} &= \frac32 \left( \frac{\sin{2 z}}{4 z^3} - \frac{\cos{2 z}}{2 z^2} \right)
	\end{align}

	\begin{figure}[htp]
		\centering
		\includegraphics[width=14cm]{SampleElectricNoiseFresnelPerfectConductor}
		\caption{graphs for $\epsilon - 1$, analytic case is $i \infty$} \label{fig:SampleElectricNoiseFresnelPerfect}
	\end{figure}

	\begin{figure}[htp]
		\centering
		\includegraphics[width=14cm]{SampleElectricNoiseZetaPerfectConductor}
		\caption{Using $\zeta$ formulation, graphs for $\epsilon - 1$, analytic case is $i \infty$} \label{fig:SampleElectricNoiseZetaPerfect}
	\end{figure}

	To get a sense of what this looks like, \fref{fig:SampleElectricNoiseFresnelPerfect} shows $g_{\parallel}(z)$, calculated using the Fresnel coefficients with $\epsilon$ with large imaginary parts to represent good conductors.
	We can see that as the conductivity gets larger, the graph more closely represents the analytic solution, but for all finite conductivities the graph diverges for small $z$.
	Typically, the $z$ range of interest is very small: for $z = \SI{1}{\micro\meter}$ and gigahertz frequencies, $z \sim 10^{-5}$.
	This divergent region is unsuitable for the Fresnel equation \todo{Because it diverges, maybe need to clarify?}.

	In \fref{fig:SampleElectricNoiseZetaPerfect} we can see that using the surface impedance forms displays the same relationships for constant $\epsilon$, which is a good sanity check.
	We do run into issues now with nested integrals, which is why it's quicker to create list plots which can be easily parallelised.
	That graph is generated with a Mathematica ListPlot which is then joined\todo{Show mesh or something?}.

	\subsection{Non-local calculations for lead} \label{subsec:sample:namresult}

	For realistic values, we'll start by looking at the slightly simpler $g_\perp = \frac32 \Re \int_0^{+\infty} \dd{u} \frac{u^3}{v} e^{2 i z v} r_p(u)$.

	We can look at how this calculation can be done for constants where $\sigma_N = \SI{e16}{\per\second}$, $\omega = \SI{e9}{\per\second}$, $\tau = \SI{e-14}{\per\second}$, $T = \SI{.8e11}{\per\second}$, $T_C = \SI{1e11}{\per\second}$ and $v_F = \SI{2e6}{\m\per\s}$.
	Using $\Delta \approx 3.06 \sqrt{Tc(Tc - T)}$, we get $\Delta \approx \SI{1.36e11}{\per\second}$.

	In the reduced units we'll use to calculate (notes on conductivity have definitions) $\Sigma$, we get $\xi = .0073$, $\nu = 730$, and $t = 0.58$.
	We will eventually be integrating over $u$, which involves a calculation like $\sigma\left(q = u \frac{\omega}{c}\right)$, which then involves $\Sigma\left(\kappa = q \frac{v_F}{\Delta}\right))$.
	Effectively, we'll be finding $\Sigma\left(\frac{\omega v_F}{c \Delta} u\right)$, or $\Sigma\left(\left( 4.87 \times 10^{-5} \right) u \right)$.
	For distances of about a micron, we'll be looking at $10^{-7} < z < 10^{-4}$.
	The $u$- and $z$-dependence will be important for letting us know what approximations we can take later.

	This incidentally corresponds with $k_F = \frac{m_e v_F}{\hbar} \approx \SI{1.7e10}{\per\meter}$, or $\kappa_F = \frac{k_F v_F}{\Delta} \approx 2.5 \times 10^5$.
	We might expect this theory to break down as we approach that value, which is fortunately much larger than the typical scales for other quantities.
	In terms of $u$, our theory will break down approaching $u_F \approx 10^{10}$.

	\subsubsection{Small-\texorpdfstring{$u$}{u} Conductivity} \label{subsubsec:smallucond}

	As we'll mention below, we effectively need to characterise the integrand for $g_\perp$ for $u$ between $0$ and something larger than $10^7$.

	We can start by noticing that $\kappa$ is very small for almost all of the range of $u$, which means we can use the small-$\kappa$ approximation.
	We can look at \fref{fig:ConductivityRe} and \fref{fig:ConductivityIm} to compare the asymptotic approximation with the full Nam form.
	As we might expect, the two correspond until around $u = 10^4$, which is where $\kappa$ is no longer much less than $1$.
	The difference between the two for $u < 10^4$ is most likely because of issues with the numerical integral in the Nam form, where the cancellations are potentially trickier for Mathematica to handle.
	This tells us where our crossover point for switching to the asymptotic form.\todo{show derivation for where the high-u asymptotic form comes from}

	\begin{figure}[htp]
		\centering
		\includegraphics[width=14cm]{ConductivityZone3Real}
		\caption{Real part of $\Sigma$(u)} \label{fig:ConductivityRe}
	\end{figure}

	\begin{figure}[htp]
		\centering
		\includegraphics[width=14cm]{ConductivityZone3Imag}
		\caption{Imaginary part of $\Sigma$(u)} \label{fig:ConductivityIm}
	\end{figure}

	\subsubsection{Breaking noise integral into regions} \label{subsubsec:noiseregions}
	We can get the most mileage out of our asymptotic forms by looking at different regions of the $g_\perp$ integral:

	To start with, we can look at the region from $0 < u < 1$.
	Here, $v = \sqrt{1 - u^2}$, and
	\begin{align}
		g_{\perp, u < 1} &= \frac32 \Re \int_0^{1} \dd{u} \frac{u^3}{\sqrt{1 - u^2}} e^{2 i z \sqrt{1 - u^2}} r_p(u) \\
			&= \frac32 \Re \int_0^{1} \dd{u} \frac{u^3}{\sqrt{1 - u^2}} r_p(u)
	\end{align}

\end{document}
