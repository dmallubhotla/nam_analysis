\providecommand{\main}{..}
\documentclass[../main.tex]{subfiles}

\title{Summary of Nam calculation method}
\author{\begin{telugu}హృదయ్ దీపక్ మల్లుభొట్ల\end{telugu}}
% want empty date
\predate{}
\date{}
\postdate{}

% Document
% !TeX spellcheck = en_GB
\begin{document}

	\graphicspath{{\main/figures/}}

	\onlyinsubfile{\maketitle}

	\section{Summary of the method we're using to calculate \texorpdfstring{$T_1$}{T1}} \label{sec:overallsummary:norm}

	This serves as a checkpoint in these notes to summarise the method we've used in calculating $T_{1}$ for the case where we use the dielectric function in~\cite{Nam1967}.

	\begin{enumerate}
		\item Use $T$ and $T_c$ to find a value for $\Delta$, using the expression $\Delta = 3.06 \sqrt{T_c(T_c - T)}$.
		\item Find the dimensionless values $\xi = \flatfrac{\omega}{\Delta}$, $\nu = \flatfrac{1}{(\tau \Delta)}$,  $t = \flatfrac{T}{\Delta}$ and $A = \flatfrac{\omega v_F}{(c \Delta)}$.
		\item Generally, $\epsilon(u) = 1 + 4 \pi i \frac{\sigma_N}{\omega} \Sigma(\xi, A u, \tau, t)$.
		Use the asymptotic expression for $\Sigma$ as $u \rightarrow 0$.
		Then,
		\begin{equation}
			-a + ib = 4 \pi i \frac{\sigma_N}{\omega} \Sigma(A u \rightarrow 0),
		\end{equation}
		from which $a$ and $b$ can be extracted by taking the real and imaginary parts.
		\item Similarly, for large $\kappa = A u$,
		\begin{align}
			\frac{-c + i d}{u} &= 4 \pi i \frac{\sigma_N}{\omega} \Sigma(\kappa > u_c) \\
			- c + i d &= 4 \pi i \frac{\sigma_N}{\omega} \Sigma(\kappa > u_c) u \\
			- c + i d &= 4 \pi i \frac{\sigma_N}{\omega} \Sigma(\kappa > u_c) \frac{\kappa}{A}
		\end{align}
		If we want to determine $c$ and $d$, we can simply increase $\kappa$ until this expression stops varying greatly, which in practice will be determined by $\tau$ and $v_F$ (as should be expected).
		\item We can determine an interpolation constant by finding $u_c$ such that
		\begin{align}
			-a + ib &= 1 + \frac{-c + id}{u_c} \\
			u_c &= \frac{-c + id}{-(a + 1) + ib}.
		\end{align}
		\item With $a$, $b$, $c$, $d$, and $u_c$, we now have enough to define our asymptotic expression
		\begin{equation}
			\epsilon(u, u_{uc}) =
			\begin{cases}
				-a + i b & u < u_c \\
				1 + \frac{-c + i d}{u} &  u_{uc} > u \geq u_c \\
				1 & u > u_{uc}
			\end{cases}.
		\end{equation}
		\item Use this expression to find $r_p(u)$ via
		\begin{equation}
			r_p(u) = \frac{1 - z(u)}{1 + z(u)},
		\end{equation}
		with
		\begin{equation}
			z(u) = \frac{2}{\pi} \int_{y_l}^{\infty} \dd{y} \frac{u}{u^2 + y^2 + (c + i d)\sqrt{u^2 + y^2}},
		\end{equation}
		with $y_l$ essentially arbitrary around the order of $u_c$.
		\item $z(u)$ will be on the order of
		\begin{equation}
			z \approx 1 - i \flatfrac{2 d}{\pi u}
		\end{equation}
		This means that 
		\begin{align}
			r_p(u) = \frac{-d^2 + i \pi d u}{u^2 \pi^2 + d^2},
		\end{align}
		if we ignore the upper cutoff $u_{uc}$.
		\item This is a much more computationally friendly equation to calculate $\chi$ via
		\begin{equation}
			\chi_{EE}(z) \propto \int_0^\infty \dd{u} u^2 e^{-2 u z} \Im[r_p(u)]
		\end{equation}
		and
		\begin{equation}
			T_1(z) \propto \frac{1}{\chi}.
		\end{equation}
		Specifically,
		\begin{equation}
			T_1(z) = \frac{\hbar \epsilon_0 c^{3}}{d^2 \omega^3} \frac{1}{\coth(\frac{\omega}{2 T}) \int_0^\infty \dd{u} u^2 e^{-2 u z} \Im[r_p(u)]}
		\end{equation}
		\item Now, we can insert some of our simplifications maybe?
		\begin{align}
			T_1(z) &= \frac{\hbar \epsilon_0 c^{3}}{d^2 \omega^3} \frac{1}{\coth(\frac{\omega}{2 T}) \int_0^\infty \dd{u} u^2 e^{-2 u z} \Im[r_p(u)]} \\
			T_1(z) &= \frac{\hbar \epsilon_0 c^{3}}{d^2 \omega^3} \frac{1}{\coth(\frac{\omega}{2 T}) \int_0^\infty \dd{u} u^2 e^{-2 u z} \frac{\pi d u}{u^2 \pi^2 + d^2}}
		\end{align}
	\end{enumerate}
	This lets us check the scale of all the remaining dimensionful constants we have, and this should be effective as long as $z \leq 1$.
	(In particular, that's important because we assume $u \geq 1$ in some simplifying steps, and $z u$ becomes the factor that determines the behaviour of the exponential.)
	For these units, $z$ is measured in terms of $\lambda = \flatfrac{\omega}{c}$, which is a far longer length scale than anything we might be interested in, so having this expression for $z \leq 1$ should always be expected to be valid enough in the areas we care about.
	This expression is also the case where we assume the cutoff momentum $u_{uc}$ is also greater than $\flatfrac{1}{z}$, so that we can effectively drop it.
	It's perhaps a less justifiable assumption, in that a good assumption would be that $u_{uc}$ reflects something around the Fermi momentum, which might actually be around $\flatfrac{1}{z}$.

\end{document}
