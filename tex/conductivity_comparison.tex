\providecommand{\main}{..}
\documentclass[../main.tex]{subfiles}

\title{Conductivity Comparison}
\author{\begin{telugu}హృదయ్ దీపక్ మల్లుభొట్ల\end{telugu}}
% want empty date
\predate{}
\date{}
\postdate{}


% Document
% !TeX spellcheck = en_GB
\begin{document}

	\graphicspath{{\main/figures/}}
	\onlyinsubfile{\maketitle}

	\section{Comparing SC noise to N dielectric constants} \label{sec:conductivitycompare:intro}

	In \fref{fig:1}, \fref{fig:2}, \fref{fig:3}, \fref{fig:4}, \fref{fig:5} and  \fref{fig:6}, plotted $\epsilon(q, \omega)$ against $q$ at $T = .8 T_c$ listed in the plot.
	They show real and imaginary parts of $\epsilon$ against the wavevector $q$.

	Broadly speaking, the imaginary parts of $\epsilon$ match up fairly well for small $q$, whereas the real parts do not.
	The large $q$ behaviour is very different qualitatively, going as $\frac{1}{q}$ in the anomalous skin effect-based calculations, rather than the $\frac{1}{q^2}$ of Lindhard.
	Note that \fref{fig:loc1}, which plots $\Im[\epsilon]$ against $\omega$, shows what we would expect: low frequency behaviour is very different in the SC case vs normal, but for frequencies higher than $2 \Delta$, the response is the same.
	The real parts of $\epsilon$ are very different across the board.
	One thing to investigate is the relationship here is whether we should be comparing against Lindhard or against the normal state anomalous skin effect, because that's the set of assumptions used in the derivations for Nam.
	(Compare Nam\autocite{Nam1967} to the set of assumptions discussed by Pippard\autocite{Pippard}, for example).
	\begin{figure}[htp]
		\centering
		\includegraphics[width=12cm]{Cond1}
		\caption{$\Im \epsilon(q, \omega = 10^9)$} \label{fig:1}
	\end{figure}
	\begin{figure}[htp]
		\centering
		\includegraphics[width=12cm]{Cond2}
		\caption{$\Im \epsilon(q, \omega = 10^{12})$} \label{fig:2}
	\end{figure}
	\begin{figure}[htp]
		\centering
		\includegraphics[width=12cm]{Cond3}
		\caption{$\Im \epsilon(q, \omega = 10^6)$} \label{fig:3}
	\end{figure}
	\begin{figure}[htp]
		\centering
		\includegraphics[width=12cm]{Cond4}
		\caption{$\Re \epsilon(q, \omega = 10^9)$} \label{fig:4}
	\end{figure}
	\begin{figure}[htp]
		\centering
		\includegraphics[width=12cm]{Cond5}
		\caption{$\Re \epsilon(q, \omega = 10^{12})$} \label{fig:5}
	\end{figure}
	\begin{figure}[htp]
		\centering
		\includegraphics[width=12cm]{Cond6}
		\caption{$\Re \epsilon(q, \omega = 10^6)$} \label{fig:6}
	\end{figure}

	\begin{figure}[htp]
		\centering
		\includegraphics[width=12cm]{local1}
		\caption{$\Im(\epsilon(q = 0, \omega))$, $T = .1 T_c$} \label{fig:loc1}
	\end{figure}
	\begin{figure}[htp]
		\centering
		\includegraphics[width=12cm]{local2}
		\caption{$\Re(\epsilon(q = 0, \omega))$, $T = .1 T_c$} \label{fig:loc2}
	\end{figure}

%	\newpage
%	\listoftodos
%	\newpage
%	\printbibliography

\end{document}
