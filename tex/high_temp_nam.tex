\documentclass[../main.tex]{subfiles}

\title{High-Temperature Nam calculations}
\author{\begin{telugu}హృదయ్ దీపక్ మల్లుభొట్ల\end{telugu}}
% want empty date
\predate{}
\date{}
\postdate{}

% Document
% !TeX spellcheck = en_GB
\begin{document}



	\onlyinsubfile{\maketitle}

	\section{High temperature Nam calculations} \label{sec:htn:intro}

	One potential avenue for fixing the cutoff momentum $q_{uc}$ is to try to relate the high temperature behaviour of the superconducting noise (again, calculated with~\cite{Nam1967})to the normal state (calculated with ~\cite{SolyomV3}).
	We know that the Lindhard function noise has fairly temperature independent behaviour, and flattens in a predictable way.
	For some value $q_{uc} \sim q_{\textrm{F}}$ where $q_{\textrm{F}}$ is the Fermi wavevector, the Nam function behaviour should behave similarly as $\frac{T}{T_c} \rightarrow 1$.
	We can essentially then numerically estimate the specific parameter $q_{uc}$ that minimises the difference between the Nam and Lindhard noise functions.

	The process is then:
	\begin{enumerate}
		\item Calculate $T_{1\textrm{,Lindhard}}^{E}(z)$ at some $z = z_{small}$, in the region where $T_1$ is effectively constant.
			In the code that follows, $z_{small}$ is chosen arbitrarily to be $\lambda_{\textrm{F}} = \frac{\hbar}{2 m_e \vf}$.
			For $\vf = \SI{2e6}{\m\per\s}$, this gives $\lambda_{\textrm{F}} \approx \SI{5.7e-11}{\m}$.
		\item For $T = .9999 T_c$, calculate $T_{1\textrm{,Nam}}^{E}(z = z_{small}, q_{uc}, T=.9999 T_c)$.
			Varying $q_{uc}$, numerically solve for the point $q_{minimised}$ where $T_{1\textrm{,Nam}} = T_{1\textrm{,Lindhard}}$.
		\item This $q_{minimised}$ should be on the order of $q_{\textrm{F}}$, and is then used in calculations of other values of $T$ and $z$.
	\end{enumerate}

	In \fref{fig:htn:1} the results of this process are plotted for $\omega=\SI{1}{\giga\Hz}$, for a variety of different temperatures.
	The temperature dependence of the Lindhard noise is very slight;
	the three different temperatures are only shown to justify the assertion that it's valid to compare the Nam noise to it at different temperatures.
	As the temperature increases, $T_1$ drops, as expected.
	For the parameters in \fref{fig:htn:1}, the optimum cutoff $q_{minimised} = \SI{5.79e9}{\per\meter}$, or $q_{minimised}=0.33 q_{\textrm{F}}$, which is a heartening sign.

	\begin{figure}[htp]
		\centering
		\includegraphics[width=12cm]{HighTempNam1}
		\caption{Noise values, $T_1$ in $\si{\second}$ and $z$ is in units of the vacuum wavelength (on the order of a meter)} \label{fig:htn:1}
	\end{figure}

	The same exercise is repeated for \fref{fig:htn:2} and \fref{fig:htn:3}, with different values for $\omega$.
	Here, because $T_c=\SI{100}{\giga\Hz}$ for these graphs, $\omega$ is closer to being on the same scale as $\Delta$, and we can see that the coherence time drops for relatively lower values of $T$.
	This is as expected.
	It is also perhaps a good sign that the curves follow each other quite closely, even though they're only being fixed at one value of $z$.

	\begin{figure}[htp]
		\centering
		\includegraphics[width=12cm]{HighTempNam2}
		\caption{\fref{fig:htn:1} again, but now with $\omega = \SI{10}{\giga\Hz}$} \label{fig:htn:2}
	\end{figure}
	\begin{figure}[htp]
		\centering
		\includegraphics[width=12cm]{HighTempNam3}
		\caption{\fref{fig:htn:1} again, but now with $\omega = \SI{100}{\giga\Hz}$, so that $\omega \sim T_c$} \label{fig:htn:3}
	\end{figure}

	Somewhat worryingly, \fref{fig:htn:3} shows points where the superconducting coherence time dips below the Lindhard.\todo{investigate?}
	There are a wide variety of reasons that a certain degree of numerical discrepancy should be expected, not the least of which being that there are multiple different approximations being used that may not guarantee that the superconducting $T_1$ stay lower than the normal state.

	The other potentially concerning feature is that the length scales don't seem to be particularly easily defined.
	For different temperatures, we see different length scales determining the flattening of the superconducting coherence time.
	It corresponds to neither the coherence length $\xi = \frac{\vf}{\pi \Delta}$, nor to the penetration depth.\todo{investigate}

	\subsection{Graphs within range} \label{subsec:htn:smallergraphs}

	In \fref{fig:htn:1GHz}, \fref{fig:htn:10GHz} and \fref{fig:htn:100GHz} we repeat the earlier graphs, but narrow the x-axis to look only at the $z$ values of interest.

	\begin{figure}[htp]
		\centering
		\includegraphics[width=12cm]{HighTempNam4}
		\caption{Comparison for different temperatures, with $z$ in units of $\mu \lambda$, $T_1$ in $\si{\nano\second}$ and $\omega = \SI{1e9}{\Hz}$} \label{fig:htn:1GHz}
	\end{figure}
	\begin{figure}[htp]
		\centering
		\includegraphics[width=12cm]{HighTempNam5}
		\caption{Comparison for different temperatures, with $z$ in units of $\mu \lambda$, $T_1$ in $\si{\nano\second}$ and $\omega = \SI{1e10}{\Hz}$} \label{fig:htn:10GHz}
	\end{figure}
	\begin{figure}[htp]
		\centering
		\includegraphics[width=12cm]{HighTempNam6}
		\caption{Comparison for different temperatures, with $z$ in units of $\mu \lambda$, $T_1$ in $\si{\nano\second}$ and $\omega = \SI{1e11}{\Hz}$} \label{fig:htn:100GHz}
	\end{figure}


\end{document}
