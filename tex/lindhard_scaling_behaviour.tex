\providecommand{\main}{..}
\documentclass[../main.tex]{subfiles}

\title{Scaling behaviour of Lindhard noise}
\author{\begin{telugu}హృదయ్ దీపక్ మల్లుభొట్ల\end{telugu}}
% want empty date
\predate{}
\date{}
\postdate{}

% Document
% !TeX spellcheck = en_GB
\begin{document}

	\graphicspath{{\main/figures/}}

	\onlyinsubfile{\maketitle}

	\section{Scale-dependent behaviour of Lindhard noise} \label{sec:lsb:intro}

	It's interesting to look at the scaling behaviour of the noise from a Lindhard metal, as shown in \fref{fig:lsb:scale1}.
	Lindhard function comes from~\cite{SolyomV3}, from which noise is calculated.

	There are a few things to note:
	\begin{itemize}
		\item The very low $z$ behaviour is reflective of the convergence of the Lindhard noise as $z \rightarrow 0$.
		\item There's a pronounced region of $T_1 \propto z^3$ behaviour in \fref{fig:lsb:scale1}.
			This is expected, and corresponds to the analytically simple $z^3$ behaviour for the local dielectric models.
		\item At larger $z$, there's a region of $z^2$ behaviour.
			For the superconducting case, we only see the quadratic scaling, so this is where the Nam case follows the Lindhard.\todo{Is there an obvious reason why this should be the case, perhaps the $k^{-2}$ vs $k^{-1}$ dependence in $\epsilon$}
	\end{itemize}

	One potential avenue to look at would be to think of these as inverse field sources effectively, so that maybe within a certain distance range there's a dipole-esque response, but far enough away it looks like a point source.\todo{Any merit to this?}

	In \fref{fig:lsb:scale2}, there are plots that show this scaling behaviour for multiple values of the collision time $\tau$.
	These curves are plotted with $z$ in units of the mean free path $\ell_1 = \tau_1 \vf$, for $\tau_1$ as the typical collision time used earlier for lead.
	Around the mean free path the behaviour starts to mimic the local cubic behaviour, exactly as we'd expect.
	This seems to be true even as we vary $\tau$;
	the point where the initial flat curve becomes cubic seems to follow $\ell$ nicely.
	Less expected is potentially the way that $\tau$ affects the transition from cubic to quadratic behaviour. \todo{Why such a weirdly long length scale for that?}
	Slightly more detail is visible in \fref{fig:lsb:scaleInflection1} and \fref{fig:lsb:scaleInflection2}.

	\begin{figure}[htp]
		\centering
		\includegraphics[width=12cm]{LSB1}
		\caption{$\frac{\ln{T_{1}^{(E)}(z)}}{\ln{z})}$, with $z$ in units of $\tau \vf$.}\label{fig:lsb:scale1}
	\end{figure}

	\begin{figure}[htp]
		\centering
		\includegraphics[width=12cm]{LSB2}
		\caption{$\frac{\ln{T_{1}^{(E)}(z)}}{\ln{z})}$, with $z$ in units of $\tau_1 \vf$ for fixed $\tau_1=\SI{1e-14}{\second}$.
		Different $\tau$ are plotted with respect to $\tau_1$.}\label{fig:lsb:scale2}
	\end{figure}

	\begin{figure}[htp]
		\centering
		\includegraphics[width=12cm]{LSB2Smaller}
		\caption{Detail from \fref{fig:lsb:scale2} around $\tau \vf$}\label{fig:lsb:scaleInflection1}
	\end{figure}

	\begin{figure}[htp]
		\centering
		\includegraphics[width=12cm]{LSB2CubicToQuadraticRegion}
		\caption{Detail from \fref{fig:lsb:scale2} around cubic to quadratic transition}\label{fig:lsb:scaleInflection2}
	\end{figure}

	\begin{figure}[htp]
		\centering
		\includegraphics[width=12cm]{LSB2_2}
		\caption{Original $T_1(z)$ for Lindhard case with $z$ in units of $\tau_1 \vf$ for fixed $\tau_1=\SI{1e-14}{\second}$.
		Different $\tau$ are plotted with respect to $\tau_1$.}\label{fig:lsb:scale2.2}
	\end{figure}

	\section{Nam scaling} \label{sec:lsb:nam}

	The same results from \fref{fig:lsb:scale1} are plotted in \fref{fig:nsb:scale1} against the values obtained for the scaling behaviour from the Nam calculation.\todo{Add more detail and discuss the quadratic part vs cubic}

	\begin{figure}[htp]
		\centering
		\includegraphics[width=12cm]{NSB1}
		\caption{$\frac{\ln{T_{1}^{(E)}(z)}}{\ln{z})}$, with $z$ in units of $\tau \vf$ for the superconducting case.}\label{fig:nsb:scale1}
	\end{figure}

\end{document}
