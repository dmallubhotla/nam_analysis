\providecommand{\main}{..}
\documentclass[../main.tex]{subfiles}

\title{Scaling behaviour of Lindhard noise}
\author{\begin{telugu}హృదయ్ దీపక్ మల్లుభొట్ల\end{telugu}}
% want empty date
\predate{}
\date{}
\postdate{}

% Document
% !TeX spellcheck = en_GB
\begin{document}

	\graphicspath{{\main/figures/}}

	\onlyinsubfile{\maketitle}

	\section{Scale-dependent behaviour of Lindhard noise} \label{sec:lsb:intro}

	It's interesting to look at the scaling behaviour of the noise from a Lindhard metal, as shown in \fref{fig:lsb:scale1}.
	Lindhard function comes from~\cite{SolyomV3}, from which noise is calculated.

	There are a few things to note:
	\begin{itemize}
		\item The very low $z$ behaviour is reflective of the convergence of the Lindhard noise as $z \rightarrow 0$.
		\item There's a pronounced region of $T_1 \propto z^3$ behaviour in \fref{fig:lsb:scale1}.
			This is expected, and corresponds to the analytically simple $z^3$ behaviour for the local dielectric models.
		\item At larger $z$, there's a region of $z^2$ behaviour.
			For the superconducting case, we only see the quadratic scaling, so this is where the Nam case follows the Lindhard.\todo{Is there an obvious reason why this should be the case, perhaps the $k^{-2}$ vs $k^{-1}$ dependence in $\epsilon$}
	\end{itemize}

	\begin{figure}[htp]
		\centering
		\includegraphics[width=12cm]{LSB1}
		\caption{$\frac{\ln{T_{1}^{(E)}(z)}}{\ln{z})}$, with $z$ in units of $\flatfrac{\omega}{c}$.}\label{fig:lsb:scale1}
	\end{figure}

\end{document}
