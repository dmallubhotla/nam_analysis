\providecommand{\main}{..}
\documentclass[../main.tex]{subfiles}

\title{Low $z$ normalisation discussion}
\author{\begin{telugu}హృదయ్ దీపక్ మల్లుభొట్ల\end{telugu}}
% want empty date
\predate{}
\date{}
\postdate{}

% Document
% !TeX spellcheck = en_GB
\begin{document}

	\graphicspath{{\main/figures/}}

	\onlyinsubfile{\maketitle}

	\section{Verifying normalisation for low \texorpdfstring{$z$}{z}} \label{sec:lowz:norm}

	The distance from the surface $z$ parameterises the maximum value of $u$ that affects the noise integral.
	Essentially, the integrand of the noise integral is the $u^2 \Im r_p$ function which goes as $\flatfrac{1}{u}$ until $u_{uc}$, at which point it becomes zero.
	Clearly, if $z < u_{uc}$, then there will be no further contributions to the integral and the noise will flatten off (as expected).
	Conversely, changing $u_{uc}$ merely should allow more of the underlying integrand to be added to the total before cutting off the sum.
	This means that there would then be more noise, which would flatten off for even smaller $z$, which is exactly what our figures show.
	Thus, our curves behave as expected.

\end{document}
