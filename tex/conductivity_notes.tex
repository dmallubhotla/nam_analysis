\providecommand{\main}{..}
\documentclass[../main.tex]{subfiles}

\title{Conductivity Notes}
\author{\begin{telugu}హృదయ్ దీపక్ మల్లుభొట్ల\end{telugu}}
% want empty date
\predate{}
\date{}
\postdate{}

% Document
% !TeX spellcheck = en_GB
\begin{document}

\graphicspath{{\main/figures/}}

\onlyinsubfile{\maketitle}
\section{Nam form of conductivity} \label{sec:NamForm}

Notes on the Nam conductivity function.\supercite{Nam1967}
We'll want to do the calculations dimensionlessly.
To remove units, we'll want to represent all the various quantities in terms of $\Delta$:
\begin{align}
	\xi &= \frac{\omega}{\Delta} \\
	\xi' &= \frac{\omega'}{\Delta} \\
	\nu &= \frac{1}{\tau \Delta} \\
	\kappa &= \frac{q v_0}{\Delta} \\
	t &= \frac{T}{\Delta} \\
	\sigma_0 &= \frac{n e^2}{m \Delta}
\end{align}

From Nam, we have
\begin{equation}
	\sigma(\kappa, \xi) = -i \frac{3 \sigma_0}{4} \frac{1}{\xi}\left[\int_{1 - \xi}^{1}\dd{\xi} \tanh(\frac{\xi + \xi'}{2 t}) I_1 + \int_{1}^{\infty} \dd{\xi'} \left( \tanh(\frac{\xi + \xi'}{2t}) I_1  - \tanh(\frac{\xi'}{2t})I_2 \right) \right]
\end{equation}
with
\begin{align}
	I_1 &= F(\kappa, \Re[\sqrt{(\xi + \xi')^2 - 1} - \sqrt{\xi'^2 - 1}]) (g + 1) \nonumber\\
	&\quad + F(\kappa, \Re[-\sqrt{(\xi + \xi')^2 - 1} - \sqrt{\xi'^2 - 1}]) (g - 1) \\
	I_2 &= F(\kappa, \Re[\sqrt{(\xi + \xi')^2 - 1} - \sqrt{\xi'^2 - 1}]) (g + 1) \nonumber\\
	&\quad + F(\kappa, \Re[\sqrt{(\xi + \xi')^2 - 1} + \sqrt{\xi'^2 - 1}]) (g - 1) \\
	F(\kappa, E) &= \frac{1}{\kappa} \left[2 S(E) + (1 - S(E)^2)\ln(\frac{S(E) + 1}{S(E) - 1})\right]  \\
	S(\kappa, E) &= \frac{1}{\kappa} \left(E - i \left(\Im[\sqrt{(\xi + \xi')^2 - 1} + \sqrt{\xi'^2 - 1}] + 2 \nu \right) \right) \\
	g  &= \frac{\xi' \left( \xi + \xi'\right) + 1}{\sqrt{\xi'^2 - 1}\sqrt{(\xi + \xi')^2 - 1}}
\end{align}

\subsection{Comparing to normal conductivity} \label{subsec:NamForm.NormalConductivity}
We can also compare this to the normal DC conductivity: $\sigma_N = \frac{n e^2 \tau}{m}$:\todo{$\sigma_N \sim 10^{16}$? Check.}
\begin{align}
	\sigma_N &= \frac{n e^2 \tau}{m} \\
	&= \frac{n e^2}{m \Delta} \tau \Delta \\
	&= \sigma_0 \frac{1}{\nu}
\end{align}

This, we can find the ratio $\Sigma = \frac{\sigma}{\sigma_N} = \frac{\sigma \nu}{\sigma_0}$,
\begin{equation}
	\Sigma(\kappa, \xi) = -i \frac{3}{4} \frac{\nu}{\xi}\left[\int_{1 - \xi}^{1}\dd{\xi} \tanh(\frac{\xi + \xi'}{2 t}) I_1 + \int_{1}^{\infty} \dd{\xi'} \left( \tanh(\frac{\xi + \xi'}{2t}) I_1  - \tanh(\frac{\xi'}{2t})I_2 \right) \right]
\end{equation}


\subsection{Verifying small $\kappa$ dependence} \label{subsec:NamForm.SmallK}
We should expect that the conductivity reaches a finite value as $\kappa \rightarrow 0$.
To verify this, we'll want to actually take that limit.
All of the dependence on momentum comes in through the function $F$ and $S$, so we can begin by writing $S$ as $S = \frac{\eta}{\kappa}$, which means that
\begin{align}
	F &= \frac{1}{\kappa} \left[2 S(E) + (1 - S(E)^2)\ln(\frac{S(E) + 1}{S(E) - 1})\right] \\
	F &= \frac{1}{\kappa} \left[2 \frac{\eta}{\kappa} + (1 - \frac{\eta^2}{\kappa^2})\ln(\frac{\frac{\eta}{\kappa} + 1}{\frac{\eta}{\kappa} - 1})\right] \label{eq:NamForm:FPreLnExpand}
\end{align}

We can then expand out the log term:
\begin{align}
	\ln(\frac{\frac{\eta}{\kappa} + 1}{\frac{\eta}{\kappa} - 1}) &= \ln(\frac{\eta + \kappa}{\eta - \kappa}) \\
	&= 2 \frac{\kappa}{\eta} + \frac23 \left(\frac{\kappa}{\eta}\right)^3 + \frac25 \left(\frac{\kappa}{\eta}\right)^5 + \mathcal{O}\left(\left(\frac{\kappa}{\eta}\right)^7\right)
\end{align}

Plugging the first three terms into~\eqref{eq:NamForm:FPreLnExpand} gives us
\begin{align}
	F &= \frac{1}{\kappa} \left[2 \frac{\eta}{\kappa} + (1 - \frac{\eta^2}{\kappa^2})\ln(\frac{\frac{\eta}{\kappa} + 1}{\frac{\eta}{\kappa} - 1})\right] \\
	&= \frac{1}{\kappa} \left[2 \frac{\eta}{\kappa} + (1 - \frac{\eta^2}{\kappa^2})\left(2 \frac{\kappa}{\eta} + \frac23 \left(\frac{\kappa}{\eta}\right)^3 + \frac25 \left(\frac{\kappa}{\eta}\right)^5 \right)\right] \\
	&= \frac{1}{\kappa} \left[2 \frac{\eta}{\kappa} + 2\frac{\kappa}{\eta} - 2 \frac{\eta}{\kappa} + \frac23 \frac{\kappa^3}{\eta^3} - \frac23 \frac{\kappa}{\eta} -  \frac25 \frac{\kappa^3}{\eta^3}\right] \\
	&= \frac{1}{\kappa} \left[\frac43\frac{\kappa}{\eta} + \frac{4}{15} \frac{\kappa^3}{\eta^3}\right] \\
	&= \frac43 \frac{1}{\eta} + \frac{4}{15} \frac{1}{\eta^3} \kappa^2
\end{align}
Here we dropped the second leading term in $\kappa^3$ before simplifying, to find that $F$ does indeed approximate a constant value as $\kappa \rightarrow 0$.

\section{Graphing the conductivity} \label{sec:NamForm.Graph}

One way we can verify that our code behaves as desired is to check that graphs of the conductivity show features that we would expect.
For all of our graphs, the easiest to visualise is the impurity parameter.
For a longer mean free path, we can see a higher conductivity, as expected.
As mentioned above, because both the superconducting and the normal conductivity depend on scattering time, graphs of $\Sigma$ will reflect both dependencies.
This can be seen in \fref{fig:ReT0K1vsWmultV}, which shows for instance that the large frequency dependence of $\Sigma$ appears proportional to $\nu$, which reflects a $\nu^{-1}$ dependence in the normal conductivity.
\begin{figure}[htp]
	\centering
	\includegraphics[width=14cm]{ReT0K1vsWmultV}
	\caption{$\Re[\Sigma(\xi, \kappa =1)],$ at $t = 0$, showing the gap for $\xi < 2$.} \label{fig:ReT0K1vsWmultV}
\end{figure}
All other figures are at $\nu = 1$.

\subsection{$\xi$ dependence} \label{subsec:NamForm.Graph.OmegaDep}
To begin with, we can look at the simplest case for $\kappa = 0$ and $t = 0$, as graphed in \fref{fig:ReT0V1vsWmultK} and \fref{fig:ImT0V1vsWmultK}.
\begin{figure}[htp]
	\centering
	\includegraphics[width=14cm]{ReT0V1vsWmultK}
	\caption{$\Re[\Sigma(\xi, \kappa)],$ at $t = 0$} \label{fig:ReT0V1vsWmultK}
\end{figure}
\begin{figure}[htp]
	\centering
	\includegraphics[width=14cm]{ImT0V1vsWmultK}
	\caption{$\Im[\Sigma(\xi, \kappa)],$ at $t = 0$} \label{fig:ImT0V1vsWmultK}
\end{figure}
\begin{figure}[htp]
	\centering
	\includegraphics[width=14cm]{ReT1V1vsWmultK}
	\caption{$\Re[\Sigma(\xi, \kappa)],$ at $t = 1$, showing conductivity in forbidden zone} \label{fig:ReT1V1vsWmultK}
\end{figure}
\begin{figure}[htp]
	\centering
	\includegraphics[width=14cm]{ImT1V1vsWmultK}
	\caption{$\Im[\Sigma(\xi, \kappa)],$ at $t = 1$} \label{fig:ImT1V1vsWmultK}
\end{figure}
As expected, we see that for $\xi < 2$ that there is no real conductivity.
This makes sense;
there are no available states in the gap for conductivity to occur. \todo{maybe explicitly show where this comes from in the Nam equation?}
For non-zero temperatures, we see in \fref{fig:ReT1V1vsWmultK} that there are low energy excitations, leading to a cusp in the graph and a non-zero real part in the gap.

Also, for $\kappa \rightarrow 0$ we expect that we should be in the local limit.
This was explored in the Zimmermann paper\supercite{Zimmermann1991}, and we can compare that our graphs here for $\kappa = 0$ have the right form. \todo{Include ref to zimmermann paper and make sure graphs match}

\subsection{$\kappa$ dependence} \label{subsec:NamForm.Graph.KDep}
\begin{figure}[htp]
	\centering
	\includegraphics[width=14cm]{ReT0V1vsKmultW}
	\caption{$\Re[\Sigma(\xi, \kappa)],$ at $t = 0$} \label{fig:ReT0V1vsKmultW}
\end{figure}
\begin{figure}[htp]
	\centering
	\includegraphics[width=14cm]{ImT0V1vsKmultW}
	\caption{$\Im[\Sigma(\xi, \kappa)],$ at $t = 0$} \label{fig:ImT0V1vsKmultW}
\end{figure}

As expected, we can see that the behaviour as a function of $\kappa$ is as expected, with the conductivity becoming real as $\kappa \rightarrow 0$, but going as $\frac{1}{\kappa}$ for large $\kappa$.
In \fref{fig:ReT0V1vsKmultW} we see the expected result that the conductivity is zero for $\xi < 2$.
However, in \fref{fig:ReT1V1vsKmultW}, with a non-zero temperature there are now non-zero conductivities for all $\xi$.
The Nam form has a peak depending on $\xi$, which is something not captured by the $\flatfrac{L_{imp}}{1 + \kappa L_{imp}}$ approximation.
\begin{figure}[htp]
	\centering
	\includegraphics[width=14cm]{ReT1V1vsKmultW}
	\caption{$\Re[\Sigma(\xi, \kappa)],$ at $t = 1$} \label{fig:ReT1V1vsKmultW}
\end{figure}
\begin{figure}[htp]
	\centering
	\includegraphics[width=14cm]{ImT1V1vsKmultW}
	\caption{$\Im[\Sigma(\xi, \kappa)],$ at $t = 1$} \label{fig:ImT1V1vsKmultW}
\end{figure}

\end{document}
